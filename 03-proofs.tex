\newcommand{\lecturetitle}[1]{
  \title{01204211 Discrete Mathematics \\ #1}
  \author{Jittat Fakcharoenphol}
  \frame{\titlepage}
}

\lecturetitle{Lecture 3: Logical proofs}

\begin{frame}\frametitle{How to prove a mathematical statement}
  Given propositions $P$ and $Q$, these are a very useful logical
  equivalences (referred to as the De Morgan's Laws).

  \begin{itemize}
  \item $\neg (P\vee Q)\equiv \neg P \wedge \neg Q$
  \item $\neg (P\wedge Q)\equiv \neg P \vee \neg Q$
  \end{itemize}

  (Note that $\neg$ takes precedence over $\vee$ or $\wedge$.)

  \vspace{0.2in}
  
  How can we prove that the first statement is true?
\end{frame}

\begin{frame}\frametitle{Proof by exhaustion}
  \begin{tcolorbox}
    For any proposition $P$ and $Q$, $\neg (P\vee Q)\equiv \neg P
    \wedge \neg Q$.
  \end{tcolorbox}
  \begin{proof}
    We will prove by exhaustion.  There are 4 cases as in the truth
    table below.

    \vspace{0.1in}
    
    \begin{tabular}{|c|c||c|c|c|}
      \hline
      $P$ & $Q$ & $P\vee Q$ & $\neg(P\vee Q)$ & $\neg Q \wedge \neg P$ \\
      \hline
      $T$ & $T$ & $T$ & $F$ & $F$ \\
      $T$ & $F$ & $T$ & $F$ & $F$ \\
      $F$ & $T$ & $T$ & $F$ & $F$ \\
      $F$ & $F$ & $F$ & $T$ & $T$ \\
      \hline
    \end{tabular}

    \vspace{0.1in}

    Note that for all possible truth values of $P$ and $Q$, $\neg
    (P\vee Q)$ equals $\neg P \wedge \neg Q$.  Thus, the statement is
    true.
  \end{proof}
\end{frame}
