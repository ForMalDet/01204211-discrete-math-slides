\newcommand{\lecturetitle}[1]{
  \title{01204211 Discrete Mathematics \\ #1}
  \author{Jittat Fakcharoenphol}
  \frame{\titlepage}
}

\lecturetitle{Lecture 5: Proof techniques 2}

\begin{frame}\frametitle{Proof techniques}
  In this lecture, we will focus on two other proof techniques.
  \begin{itemize}
  \item Proofs by contradiction
  \item Proofs by cases
  \end{itemize}
\end{frame}

\begin{frame}\frametitle{Proofs by contradiction}
  We want to prove that proposition $P$ is true.  To do so, we first
  assume that $P$ is false, and show that this logically leads to a
  contradiction.  This means that it is impossible for $P$ to be
  false; hence, $P$ has to be true.  This is called a proof by
  contradiction or {\em reductio ad absurdum}.

  \begin{tcolorbox}[title=Direct proofs]
    {\bf Theorem:} \\
    $P$
    \begin{proof}
      We use prove by contradiction.
      
      Assume $\neg P$.
      
      ... (then show that $R$ and $\neg R$ follows from $\neg P$)

      This is a contradiction. Therefore, $P$ must be true.
    \end{proof}
  \end{tcolorbox}
\end{frame}

\begin{frame}\frametitle{Example 1 (1)}
  \begin{theorem}
    $\sqrt{2}$ is irrational.
  \end{theorem}
  \begin{proof}
    We prove by contradiction.  Assume that the theorem is false,
    i.e., assume that $\sqrt{2}$ is rational. \pause

    Therefore, there exists a pair of positive integers $a$ and $b$
    such that $\sqrt{2} = a/b$.  \pause Let's choose the pair $a$ and
    $b$ such that $b$ is minimum. In this case, $a$ and $b$ share no
    common factors. \pause

    Let's square both terms.  We get $2 = a^2/b^2$, or
    \[ a^2 = 2b^2. \]

    (cont. in next slide)
  \end{proof}
\end{frame}


\begin{frame}\frametitle{Example 1 (2)}
  \begin{proof}[Proof. (cont.)]
    By definition, we know that $a^2$ is an even number.  From a
    theorem from last time, we know that $a$ must also be an even
    number. \pause

    Again by definition, there exists integer $k$ such that $a = 2k$.
    We then obtain
    \[2b^2 = (2k)^2 = 4k^2,\]
    i.e., $b^2 = 2k^2$.  \pause This implies that $b^2$ is an even number.
    Again, this means that $b$ must be an even number. \pause

    \vspace{0.1in}
    \textcolor{blue}{{\bf [quick check] } Do you see that we are
      arriving at a contradiction here?}
    \vspace{0.1in}
    \pause

    (cont. in the next slide)
  \end{proof}
\end{frame}

\begin{frame}\frametitle{Example 1 (3)}
  \begin{proof}[Proof. (cont.)]
    Since $a$ and $b$ are both even numbers, they share $2$ as a
    common factor. \pause

    This contradicts the fact that we choose the pair $a$ and $b$ that
    share no common factor. \pause

    Therefore, $\sqrt{2}$ must be irrational.
  \end{proof}
\end{frame}

\begin{frame}\frametitle{Proofs by cases}
  \begin{itemize}
  \item The last proof technique that we shall discuss is closely
    related to proofs by exhaustion we tried before.
  \item Sometimes when we want to prove a statement, there are many
    possible cases.  Also, we might not know which cases are true.
  \item We might still be able to prove the statement if we can show
    that the statement is true in every case.
  \end{itemize}
\end{frame}
  
\begin{frame}\frametitle{Example 2 (1)}
  \begin{theorem}
    Suppose that I have 3 pairs of socks: one pair in gray, one pair
    in white, and one pair in black.  If I pick any 4 socks, I will
    have at least one pair of the same color.
  \end{theorem}
  \pause

  \textcolor{blue}{If we want to prove by exhaustion, we will have to
    consider all 15 cases.}
  \pause
  
  \begin{proof}
    Let's split the process of picking 4 socks into 2 steps.  First,
    pick 3 socks, then pick the last sock.

    After we pick the first 3 socks.  There are 2 possible cases:
    either I have a pair of socks with the same color, or I do not
    have such a pair.  We shall consider each case separately.

    (cont. in the next slide)
  \end{proof}
\end{frame}

\begin{frame}\frametitle{Example 2 (1)}
  \begin{proof}[Proof. (cont.)]
    \begin{itemize}
    \item
      {\bf Case 1:} {\em I have a pair of socks with the same color.} \\
      \pause
      In this case, the theorem is true.
      \pause
    \item
      {\bf Case 2:} {\em I do not have a pair of socks with the same
      color.} \\
      \pause
      In this case, since I have 3 colors and 3 socks, I must have one
      sock for each color.  Now, after we pick the last sock, whatever
      color the last one is, we have a color-matching sock in our
      first 3 socks.  Therefore, the theorem is also true in this
      case.
    \end{itemize}
    \pause

    Since these two cases cover all possibilities, we conclude that
    the theorem is true.
  \end{proof}
\end{frame}

\begin{frame}\frametitle{Proofs by cases in propositional logic}
  In propositional logic, the following describe a proof by cases.

  \begin{tcolorbox}
    \begin{tabular}{l}
      $P \vee Q \vee R$ \\
      $P\Rightarrow S$ \\
      $Q\Rightarrow S$ \\
      $R\Rightarrow S$ \\
      \hline
      $S$
    \end{tabular}
  \end{tcolorbox}
  \pause

  Sometimes, when we have 2 cases, we also see:
  \pause
  
  \begin{tcolorbox}
    \begin{tabular}{l}
      $P\vee\neg P$\\
      $P\Rightarrow S$ \\
      $\neg P \Rightarrow S$ \\
      \hline
      $S$
    \end{tabular}
  \end{tcolorbox}

  Note that we can leave $P\vee\neg P$ out, because it is always true.
\end{frame}
