\newcommand{\lecturetitle}[1]{
  \title{01204211 Discrete Mathematics \\ #1}
  \author{Jittat Fakcharoenphol}
  \frame{\titlepage}
}

\lecturetitle{Lecture 17: Primality testing (2)} 

\begin{frame}\frametitle{Efficient algorithms}
  This lecture proves that the two forms of the Fermat's Little
  Theorem are equivalent.  

  We then outline a primality testing algorithm based on the Fermat's
  Little Theorem.  This algorithm requires two subroutines for
  \begin{itemize}
  \item computing powers $a^n$,
  \item finding the greatest common divisors.
  \end{itemize}
  We will present efficient algorithms for these two problems.
\end{frame}

\begin{frame}\frametitle{Fermat's Little Theorem}
  The common form of the Fermat's Little Theorem is as follows.

  \begin{tcolorbox}[title=First form]
    \textcolor{blue}{Theorem:} If $p$ is a prime and $a$ is an integer not divisible by $p$, then $a^{p-1}\Mod p = 1$.
  \end{tcolorbox}

  However, we prove the following version in the previous lecture:

  \begin{tcolorbox}[title=Second form]
    \textcolor{blue}{Theorem:} If $p$ is a prime, then $p|a^{p}-a$, for any integer $a$.
  \end{tcolorbox}

  We shall prove that both forms are equivalent.
\end{frame}

\begin{frame}\frametitle{Basic claim}
  We need the following claim to prove the equivalence.

  \begin{tcolorbox}
    {\bf Claim 1:} If $p$ is a prime and $p| ab$ but $p\not| a$, then
    $p|b$.
  \end{tcolorbox}
  
  {\bf Proof:} Factor $ab$ into a product of primes.  Since $p$
  divides $ab$ then $p$ must be one of the factors in the prime
  factorization of $ab$.  However, since $p\not|a$, $p$ does not
  appear in $a$'s prime factorization.  This implies that $p$ must be
  from $b$'s prime factorization.  Hence, $p|b$.  $\blacksquare$

  \vspace{0.2in}

  Note that the assumption that $p$ is a prime is crucial.  For
  example, if we let $p=6$, $a=9$, and $b=2$, we can see that $6|18$,
  but $6\not|9$ or $6\not|2$.
\end{frame}

\begin{frame}\frametitle{Another form for the Fermat's Little Theorem}
  To show that both forms of the Fermat's Little Theorem are
  equivalent, we need to prove the following two claims.  (You should
  look at the claims and try to understand why proving both of them
  implies that both forms of the Fermat's Little Theorem are
  equivalent.)

  \begin{tcolorbox}[title=Second form $\Rightarrow$ first form]
    \textcolor{blue}{Claim 2:} If $p$ is a prime, $p$ does not divide an integer $a$, and $p|a^p-a$; then $a^{p-1}\Mod p = 1$.
  \end{tcolorbox}

  {\bf Proof:} Write $a^p-a = a(a^{p-1}-1)$.  Since $p|a(a^{p-1}-1)$
  but $p\not|a$, from claim 1, we have that $p|a^{p-1}-1$.  Thus, 
  \[ a^{p-1}\Mod p = 1, \]
  as required. $\blacksquare$
\end{frame}

\begin{frame} 
  \begin{tcolorbox}[title=First form $\Rightarrow$ second form]
    \textcolor{blue}{Claim 3:} Lat $p$ be a prime.  If for any integer
    $a$ not divisible by $p$, $a^{p-1}\Mod p = 1$, then for any
    integer $b$, we have that $p|b^p-b$.
  \end{tcolorbox}
  
  {\bf Proof:} There are two cases, when $p\not|b$ and $p|b$.  

  In the first case where $p\not|b$, the assumption implies that
  $p|b^{p-1}-1$; thus, $p|b(b^{p-1}-1)$, as required.

  Now consider the case when $p|b$.  Since we can write
  $b^p-b=b(b^{p-1}-1)$, we know that $p|b(b^{p-1}-1)$, because
  $p|b$. $\blacksquare$
\end{frame}

\begin{frame}\frametitle{The test: idea}
  If we want to check if $q$ is a prime, from the Fermat's Little
  Theorem, we know that when $q$ is a prime, for any integer $a$ not
  divisible by $q$, $a^{q-1}\Mod q=1$.  We can devise a test based on
  this fact:

  \begin{tcolorbox}
    {\bf PROCEDURE} FermatTest0($q$,$a$)\\
    1. Find $y = a^{q-1}\Mod q$\\
    2. {\bf if} $y\neq 1$ {\bf then}\\
    3. \ \ {\bf return} ``COMPOSITE''\\
    4. {\bf else}\\
    5. \ \ {\bf return} ``PRIME''\\
    6. {\bf endif}
  \end{tcolorbox}

  To implement this test efficiently, we need a way to find
  $a^{q-1}\Mod q$ quickly.  Recall that in our test, $q$ will be a
  very large number and an $O(q)$-time algorithm or even
  $O(\sqrt{q})$-time algorithm will not be fast enough.
\end{frame}

\begin{frame}\frametitle{Computing powers}
  If we want to find $a^n\Mod q$, a straight-forward implementation is
  to perform $n$ multiplications.  When $a$ and $q$ are $m$-bit
  numbers, this takes $O(nm^2)$ time because multiplication and
  division run in $O(m^2)$ time.

  Can we compute the power faster?

  Let's start with some example.
  \begin{itemize}
  \item How to compute $a^2$?  We can't do anything faster then just
    computing directly $a^2$.
  \item What can we compute if we know $a^2$?  Note that we can square
    $a^2$ to get $a^2\times a^2=a^4$.
  \item We can keep squaring to get $a^8=a^4\times a^4$,
    $a^{16}=a^8\times a^8$, and so on.
  \item Thus, we can compute $a^{2^k}$ with only $k$ multiplications.
  \end{itemize}

  We know how to compute the power $a^n$ when $n$ is a power of $2$.
  We still need to handle the case when the exponent $n$ is not a
  power of $2$.
\end{frame}

\begin{frame}
  To deal with general case, we can either keep all computed powers
  and multiply appropriate results to get the final answer.  For
  example, to compute $a^{20}$, we note that $a^{20}=a^{16}\cdot a^4$,
  so we first compute $a^2,a^4,a^8,a^{16}$ and multiply $a^{16}$ and
  $a^4$.

  \vspace{0.1in} A recursive algorithm below takes another approach.
  It first finds $a^{\lfloor n/2\rfloor}$, where $\lfloor x\rfloor$ is
  a floor function that returns the largest integer not greater than
  $x$ (e.g., $\lfloor 1.2\rfloor=1$ and $\lfloor 2\rfloor=2$).  It
  then squares the result and depending on the value of $n/2$,
  multiplies the result with $a$.

  \begin{tcolorbox}
    {\bf PROCEDURE} Power($a$,$n$)\\
    1. {\bf if} $n=1$ {\bf then} {\bf return} $a$ {\bf endif}\\
    2. {\bf let} $b\leftarrow$ Power($a$,$\lfloor n/2\rfloor$)\\
    3. {\bf if} $2|n$ {\bf then} \\
    4. \ \ {\bf return} $b\times b$ \\
    5. {\bf else}\\
    6. \ \ {\bf return} $b\times b\times a$ \\
    7. {\bf endif}
  \end{tcolorbox}

\end{frame}

\begin{frame}\frametitle{The number of multiplications}
  Let's analyze the number of multiplications required to compute
  $a^n$.
\end{frame}

\begin{frame}\frametitle{The running time}
\end{frame}

\begin{frame}\frametitle{The greatest common divisors}
\end{frame}

\begin{frame}\frametitle{The test: idea + more condition}
  If $q$ is a prime and $a$ is chosen so that $2\leq a\leq q-1$, we
  know that $gcd(q,a)=1$.  We shall add this condition to the test as
  well.  While this check might not be that helpful in itself when $q$
  is composite, it is crucial to our proof that the algorithm works
  (described in the next lecture).

  \begin{tcolorbox}
    {\bf PROCEDURE} FermatTest($q$,$a$)\\
    1. {\bf if} $GCD(q,a)\neq 1$ {\bf then} {\bf return} ``COMPOSITE'' {\bf endif}\\
    2. Find $y = a^{q-1}\Mod q$\\
    3. {\bf if} $y\neq 1$ {\bf then}\\
    4. \ \ {\bf return} ``COMPOSITE''\\
    5. {\bf else}\\
    6. \ \ {\bf return} ``PRIME''\\
    7. {\bf endif}
  \end{tcolorbox}
\end{frame}

\begin{frame}\frametitle{Finding the greatest common divisors}
\end{frame}

\begin{frame}\frametitle{The Euclidean algorithm}
\end{frame}

\begin{frame}\frametitle{An example}
\end{frame}

\begin{frame}\frametitle{The running time}
\end{frame}

