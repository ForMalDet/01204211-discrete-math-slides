\newcommand{\lecturetitle}[1]{
  \title{01204211 Discrete Mathematics \\ #1}
  \author{Jittat Fakcharoenphol}
  \frame{\titlepage}
}

\lecturetitle{Lecture 6: Mathematical Induction 1}

\begin{frame}\frametitle{Mathematical Induction}
  \begin{itemize}
  \item In this lecture, we will focus on how to prove properties on natural numbers. \pause
  \item For example, we may want to prove that for any integer $n\geq 1$,
    \[ \sum_{i=1}^n i = i(i+1)/2, \]
    \pause
    or for any integer $n\geq 1$,
    \[ \sum_{i=1}^n i^2 = \frac{n}{6}(n+1)(2n+1),\]
    \pause
    or ``We can pay any integer amount $x\geq 4$ baht with 2-baht
    coins and 5-baht coins.''
  \end{itemize}
\end{frame}

\begin{frame}\frametitle{Informal arguments}
  \begin{itemize}
  \item Let's try to check that $\sum_{i=1}^n i = i(i+1)/2$, for any
    integer $n\geq 1$, by experimentation.
  \item Try $n=1$: \pause LHS: $1$, \pause RHS: $1(1+1)/2 = 1$, \pause OK
  \item Try $n=2$: \pause LHS: $1+2=3$, \pause RHS: $2(2+1)/2 = 3$, \pause OK
  \item Try $n=3$: \pause LHS: $1+2+3=6$, \pause RHS: $3(3+1)/2 = 6$, \pause OK
  \item Try ... \pause
  \item With this approach, we can't actually prove this statement.
  \end{itemize}
\end{frame}
