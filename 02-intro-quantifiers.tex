\documentclass{beamer}
\usefonttheme[onlymath]{serif}
\usepackage{fancybox}
\usepackage{tcolorbox}
\begin{document}

\newcommand{\lecturetitle}[1]{
  \title{01204211 Discrete Mathematics \\ #1}
  \author{Jittat Fakcharoenphol}
  \frame{\titlepage}
}

\lecturetitle{Lecture 2: Quantifiers and proofs}

\begin{frame}\frametitle{This lecture covers:}
  \begin{itemize}
  \item More on quantifiers
  \item How to prove a proposition
  \item Basic proof techniques
  \end{itemize}
\end{frame}

\begin{frame}\frametitle{Review: Quantifiers}
  \begin{itemize}
  \item A {\em predicate} is a statement with variables, which can be
    either true or false, after all its variables are specified.
  \item If we quantify a predicate completely, the quantified
    expression now has a truth value, and it is called a quantified
    proposition.
  \item Two ways to quantify:
    \begin{itemize}
    \item {\bf Universal quantifier ($\forall$)} states that the
      quantified proposition is true when the predicate is true for
      every value of the variable in the specified set.
    \item {\bf Existential quantifier ($\exists$)} states that the
      quantified proposition is true when the predicate is true for at
      least one value of the variable in the specified set.
    \end{itemize}

  \item Quantifiers can be nested.  E.g., 
    \begin{itemize}
    \item $\forall x\forall y P(x,y)\equiv \forall x(\forall y (P(x,y)))$
    \item $\forall x\exists y P(x,y)\equiv \exists x(\forall y (P(x,y)))$
    \end{itemize}
  \end{itemize}
\end{frame}

\end{document}
