\newcommand{\lecturetitle}[1]{
  \title{01204211 Discrete Mathematics \\ #1}
  \author{Jittat Fakcharoenphol}
  \frame{\titlepage}
}

\lecturetitle{Lecture 10: Counting 2}

\begin{frame}\frametitle{Listing all subsets\footnote{This section follows section 1.3 from [LPV].}}
  \begin{itemize}
  \item From the previous lecture, we know that a set with $n$
    elements has $2^n$ subsets.
  \item Let's try to enumerate them. \pause
  \item As an example, consider set $\{a,b,c\}$ and its subsets.
  \item There are many ways of listing all 8 subsets.
    \pause
    \begin{itemize}
    \item $\emptyset$, $\{a\}$, $\{a,b\}$, $\{a,b,c\}$,
      $\{a,c\}$, $\{b\}$, $\{b,c\}$, $\{c\}$
      \pause

      Note that we treat each subset as a word in a dictionary and use
      the dictionary order.
      \pause
      
    \item $\emptyset$, $\{a\}$, $\{b\}$, $\{c\}$,
      $\{a,b\}$, $\{a,c\}$, $\{b,c\}$, $\{a,b,c\}$
      \pause

      In this case, we order by their cardinalities, then use
      dictionary ordering for subsets with the same numbers of
      elements.
    \end{itemize}
  \end{itemize}
\end{frame}

\begin{frame}\frametitle{A different representation (1)}
  \begin{itemize}
  \item There is a different representation for subsets which is
    particularly useful when listing subsets.
  \item To represent a subset of $A=\{a,b,c\}$, we consider each
    element of $A$ one-by-one in some fixed order.  If that element is
    in the subset, we write down 1, if it is not we write down 0.
  \item For examples:
    \begin{itemize}
    \item $\{a,c\}$ is represented as: \pause $101$
    \item $\{a\}$ is represented as: \pause $100$
    \item $\{b,c\}$ is represented as: \pause $011$
    \item $\{\}$ is represented as: \pause $000$
    \end{itemize}
  \end{itemize}
\end{frame}

\begin{frame}\frametitle{A different representation (2)}
  \begin{itemize}
  \item Note that we represent a subset as a string with 0's and 1's.
    You may recall that these strings can be considered as binary
    numbers.
  \item Thus, we can associate the numerical values of the
    representations with the subsets:
    \pause
    \begin{itemize}
    \item $\{a,c\}$ is rep. as: $101_2$ \pause $= 5$, \ \ \ \pause
      $\{a\}$ is rep. as: $100_2$ \pause $= 4$
    \item $\{b,c\}$ is rep. as: $011_2$ \pause $= 3$, \ \ \ \pause
      $\{\}$ is rep. as: $000_2$ \pause $= 0$
    \end{itemize}
    \pause
  \item Also, this representation can be considered backwards, i.e.,
    if we start with an integer $6$, we can write down its binary
    representation: $110_2$ and turns it into a subset $\{a,b\}$.
  \end{itemize}
\end{frame}

\begin{frame}\frametitle{A correspondence}
  Let's see a full list of correspondence between $\{0,1,2,\ldots,7\}$
  and subsets of $\{a,b,c\}$.
  \begin{itemize}
  \item $0 \leftrightarrow 000_2 \leftrightarrow \{\}$
  \item $1 \leftrightarrow 001_2 \leftrightarrow \{c\}$
  \item $2 \leftrightarrow 010_2 \leftrightarrow \{b\}$
  \item $3 \leftrightarrow 011_2 \leftrightarrow \{c,b\}$
  \item $4 \leftrightarrow 100_2 \leftrightarrow \{a\}$
  \item $5 \leftrightarrow 101_2 \leftrightarrow \{a,c\}$
  \item $6 \leftrightarrow 110_2 \leftrightarrow \{a,b\}$
  \item $7 \leftrightarrow 111_2 \leftrightarrow \{a,b,c\}$
  \end{itemize}
  \pause

  Do you notice anything interesting?
\end{frame}

\begin{frame}\frametitle{A general case}
  Similarly, we can describe a representation for each subset of a set
  $A$ with $n$ elements.  As we consider each element $a$ of $A$, we
  put $1$ if $a\in A$ and put $0$ if $a\not\in A$.
  \pause

  Each subset is represented uniquely as a string of $0$ and $1$ of
  length $n$.  Also, each string corresponds to only one subset.
  Then, we can conclude that the number of subsets equal the number of
  bit strings of length $n$.
  \pause

  How many bit strings of length $n$ are there?
  \pause

  \vspace{0.1in}

  There are $2^n$ bit strings; hence, the number of subsets is also
  $2^n$.
  \pause
  
  This is another proof of the following theorem:
  
  \begin{tcolorbox}
    {\bf \textcolor{blue}{Theorem:}} The number of subsets of a set with
    $n$ elements is $2^n$.
  \end{tcolorbox}
\end{frame}

\begin{frame}\frametitle{Two proofs}
  Why do we need two proofs of the same statement?
  \pause

  Really, it does not make a statement stronger, truer, ``more''
  correct.  But each proof usually reveals additional facts related to
  the statement.

  \begin{itemize}
  \item The first proof considers a procedure for constructing subsets.
  \item The second proof introduces a nice technique for counting.
    I.e., instead of counting subsets directly, we show that we have a
    ``special'' correspondence between subsets and binary numbers, and
    then just count the numbers.
  \end{itemize}
\end{frame}

\begin{frame}\frametitle{A bijection}
  What is so special about this correspondence?
  \pause

  \begin{itemize}
  \item For each number, there is only {\bf one} subset that
    corresponds to it.
  \item For each subset, there is only {\bf one} number that it
    corresponds to.
  \end{itemize}

  With these two properties, we can conclude that both sets have the
  same cardinality.

  \begin{tcolorbox}
    This type of correspondence is called a {\bf
      one-to-one correspondence} or {\bf bijection}.
  \end{tcolorbox}
\end{frame}

\begin{frame}\frametitle{Sequences of choices}
  Previously, when we want to count the number of bit strings of
  length $n$, we use this argument:

  \begin{tcolorbox}
    Suppose that to select an object, you have to make $k$ decisions.
    The first decision has $n_1$ choices, the second decision has
    $n_2$ choices, and so on.  More precisely, for $1\leq i\leq k$,
    the $i$-th decision has $n_i$ choices.  Then the number of ways
    you can select an object is $n_1\cdot n_2\cdots n_{k-1}\cdot n_k$.
  \end{tcolorbox}
\end{frame}

\begin{frame}\frametitle{Example 1}
  \begin{tcolorbox}
    A car license number consists of two English letters and one
    number from 1 to 9999.  How many possible license numbers are
    there?
  \end{tcolorbox}
  \vspace{2in}
\end{frame}

\begin{frame}\frametitle{Example 2}
  \begin{tcolorbox}
    10 students stand in a line.  You want to give them ice cream.
    There are 4 flavours, but you don't want to give the same flavour
    to any consecutive students.  In how many ways can you give out
    the ice cream to these students?
  \end{tcolorbox}
  \vspace{2in}
\end{frame}

\begin{frame}\frametitle{Permutations}
\end{frame}

\begin{frame}\frametitle{Counting permutations: an example}
  We want to count the number of permutations.  Let's try with a small
  example: permutations of set $\{a,b,c\}$.

  \vspace{2.5in}
\end{frame}

\begin{frame}\frametitle{Counting permutations}
\end{frame}

\begin{frame}\frametitle{Number of permutations}
  We have proved this theorem.
  \begin{tcolorbox}
    {\bf \textcolor{blue}{Theorem:}} The number of permutations of a
    set with $n$ elements is $n!$.
  \end{tcolorbox}
\end{frame}



