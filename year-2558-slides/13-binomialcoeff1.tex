\newcommand{\lecturetitle}[1]{
  \title{01204211 Discrete Mathematics \\ #1}
  \author{Jittat Fakcharoenphol}
  \frame{\titlepage}
}

\lecturetitle{Lecture 13: Binomial Coefficients (1)} 

\begin{frame}\frametitle{The binomial coefficients\footnote{This lecture mostly follows Chapter 3 of [LPV].}}
  There is a reason why the term $\binom{n}{k}$ is called the binomial
  coefficients.  In this lecture, we will discuss
  \begin{itemize}
  \item the Pascal's triangle, 
  \item the binomial theorem, and
  \item advanced counting with binomial coefficients.
  \end{itemize}
\end{frame}

\begin{frame}\frametitle{The equation}
  Last time we proved that, for $n,k>0$,
  \[\binom{n}{k} = \binom{n-1}{k-1} + \binom{n-1}{k}.\]
  \pause

  While we can prove this equation algebraically using definitions of
  binomial coefficients, proving the fact by describing the process of
  choosing $k$-subsets reveals interesting insights.  This equation
  also hints us how to compute the value of $\binom{n}{k}$ using
  values of $\binom{n}{\cdot}$'s.

  \pause
  So, let's try to do it.
\end{frame}

\begin{frame}\frametitle{The table}
  We shall use the fact that $\binom{n}{0}=1$ and $\binom{n}{k} =
  \binom{n-1}{k-1} + \binom{n-1}{k}$ to fill in the following table.

  \begin{tabular}{|r|c|c|c|c|c|c|c|}
    \hline
    $n$ & 0 & 1 & 2 & 3 & 4 & 5 & 6 \\ 
    \hline
    $0$ & 1 &&&&&&\\
    \hline
    $1$ & 1 & 1 &&&&&\\
    \hline
    $2$ & 1 & \pause 2 & 1 &&&&\\
    \hline
    \pause
    $3$ & 1 & \pause 3 & 3 & 1 &&&\\
    \hline
    \pause
    $4$ & 1 & \pause 4 & 6 & 4 & 1 &&\\
    \hline
    \pause
    $5$ & 1 & \pause 5 & 10 & 10 & 5 & 1 &\\
    \hline
    \pause
    $6$ & 1 & \pause 6 & 15 & 20 & 15 & 6 & 1 \\
    \hline
  \end{tabular}
  
  \vspace{0.1in}

  \pause You can note that the table is left-right symmetric.  This is
  true because of the fact that $\binom{n}{k} = \binom{n}{n-k}$.
\end{frame}

\begin{frame}\frametitle{The Triangle}
  If we move the numbers in the table slightly to the right, the table
  becomes the Pascal's triangle.
  \pause

  \vspace{0.1in}

  \begin{tcolorbox}
  \begin{tabular}{ccccccccccccc}
    & & & & & & 1 & & & & & & \\
    & & & & & 1 & & 1 & & & & & \\
    & & & & 1 & & 2 & & 1 & & & & \\
    & & & 1 & & 3 & & 3 & & 1 & & & \\
    & & 1 & & 4 & & 6 & & 4 & & 1 & & \\
    & 1 & & 5 & & 10 & & 10 & & 5 & & 1 & \\
    1 & & 6 & & 15 & & 20 & & 15 & & 6 & & 1 \\
    & $\vdots$ & & $\vdots$ & & & & & & & & $\vdots$ & \\
  \end{tabular}
  \end{tcolorbox}

  \vspace{0.1in}
  
  The table and the binomial coefficients have many other interesting
  properties.
\end{frame}

\begin{frame}\frametitle{Polynomial expansions}
  Let's start by looking at polynomial of the form $(x+y)^n$.  Let's
  start with small values of $n$:
  \begin{itemize}
  \item $(x+y)^1=x+y$
  \item $(x+y)^2 = \pause x^2 + 2\cdot xy + y^2$\\
  \item \pause $(x+y)^3 = \pause x^3 + 3\cdot x^2y + 3\cdot xy^2 + y^3$\\
  \item \pause $(x+y)^4 = \pause x^4 + 4\cdot x^3y + 6\cdot x^2y^2 + 4\cdot xy^3 + y^4$.
  \end{itemize}
  
  \vspace{0.1in}
  Let's focus on the coefficient of each term.  You may notice that
  terms $x^n$ and $y^n$ always have 1 as their coefficients.  {\em Why
    is that?} \pause

  Let's look further at the coefficients of terms $x^{n-1}y$.  Do you
  see any pattern in their coefficients?  {\em Can you explain why?}
\end{frame}

\begin{frame}\frametitle{Another way to look at it}
  Let's take a look at $(x+y)^4$ again.  It is

  \[ (x+y)(x+y)(x+y)(x+y). \]

  \begin{itemize}
  \item How do we get $x^4$ in the expansion?  \pause For every
    factory, you have to pick $x$.
  \item How do we get $x^3y$ in the expansion? \pause Out of the 4
    factors, you have to pick $y$ in one of the factor (or you have to
    pick $x$ in 3 of the factors).  \pause Thus there are
    $\binom{4}{3}=\binom{4}{1}$ ways to do so.
  \end{itemize}
\end{frame}

\begin{frame}\frametitle{The binomial theorem}
  \begin{tcolorbox}
    \textcolor{blue}{Theorem:} If you expand $(x+y)^n$, the
    coefficient of the term $x^ky^{n-k}$ is $\binom{n}{k}$.
  \end{tcolorbox}
  That is,
  \[ (x+y)^n = \sum_{k=0}^n \binom{n}{k} x^ky^{n-k} = \]
  \[ \binom{n}{n} x^n + \binom{n}{n-1} x^{n-1}y^1 + \binom{n}{n-2} x^{n-2}y^2 + \cdots + \binom{n}{1}xy^{n-1} + \binom{n}{0} y^n.\]
\end{frame}

\begin{frame}\frametitle{Additional applications of the binomial theorem}
  The binomial theorem can be used to prove various identities
  regarding the binomial coefficients.  For example, if we let $x=1$
  and $y=1$, we get that
  \[(1+1)^n=2^n=\binom{n}{0}+\binom{n}{1}+\cdots+\binom{n}{n-1}+\binom{n}{n}.\]

  \pause

  \vspace{0.2in}

  \begin{tcolorbox}
    {\bf Quick check.}  Can you prove that
    \[\binom{n}{0} - \binom{n}{1} + \binom{n}{2} - \binom{n}{3} + \cdots = 0.\]

    {\em Note that this statements says that the number of odd subsets
      equals the number of even subsets.}
  \end{tcolorbox}
\end{frame}

\begin{frame}\frametitle{More on counting}
  We shall see more techniques for counting when we consider the
  following problems.
  \begin{itemize}
  \item How many anagrams does the word ``KASETSARTUNIVERSITY'' have?
    (They do not have to be real English words.)
  \item How can you give out $n$ different presents to $k$ students
    when student $i$ has to get $n_i$ pieces of presents?
  \item How many ways can you distribute $n$ baht coins to $k$
    children?
  \end{itemize}
\end{frame}

\begin{frame}\frametitle{Easy anagrams}
  \begin{itemize}
  \item An anagram of a particular word is a word that uses the same
    set of alphabets.  For example, the anagrams of $ADD$ are $ADD$,
    $DAD$, and $DDA$. \pause
  \item How many anagrams does ``$ABCD$'' have? \pause
    \begin{itemize}
    \item $4!$, because every permutation of A B C or D is a different
      anagram. \pause
    \end{itemize}
  \end{itemize}
\end{frame}

\begin{frame}\frametitle{Harder anagrams}
  \begin{itemize}
  \item How many anagrams does ``$ABCC$'' have? Is it $4!$ ? \pause
    \begin{itemize}
    \item This time we have to be careful because the answer of $4!$
      is too large as it over counts many anagrams, i.e., it
      ``distinguishes'' the two $C$'s. \pause
    \item Let's try to be concrete. How many times does ``$CABC$'' get
      counted in $4!$? \pause
    \item If we treat two $C$'s differently as $C_1$ and $C_2$, we can
      see that $CABC$ is counted twice as $C_1ABC_2$ and $C_2ABC_1$.
      This is true for any anagram of $ABCC$.  \pause
    \item Since each anagram is counted in $4!$ twice, the number of
      anagrams is $4! / 2 = 4\cdot 3 = 12$.
    \end{itemize}
  \end{itemize}
\end{frame}

\begin{frame}\frametitle{General anagrams}
  \begin{tcolorbox}
    Let's try to use the same approach to count the anagram of
    $HELLOWORLD$. (It has 3 $L$'s, 2 $O$'s, $H$, $E$, $W$, $R$, and
    $D$.)
  \end{tcolorbox}
  
  \pause
  \vspace{0.2in}
  
  The number of permutation of alphabets in $HELLOWORLD$, treating
  each character differently is $10!$.  However, each anagram is
  counted for $3!2!$ times because of the 3 copies of $L$ and the 2
  copies of $O$.  Therefore, the number of anagrams is
  \[
  \frac{10!}{3!2!}.
  \]
\end{frame}

\begin{frame}\frametitle{Distributing presents}
  \begin{tcolorbox}
    I have $9$ different presents.  I want to give them to $3$
    students: A, B, and C.  I want to give each student $3$ presents.
    In how many ways can I do it?
  \end{tcolorbox}
  
  \pause

  {\small
    \begin{itemize}
    \item Let's think about the process of distributing the
      presents. \pause We can first let A choose $3$ presents, then B
      chooses the next $3$ presents, and C chooses the last $3$
      presents. \pause If we distinguish the order which each child
      chooses the presents, then there are $9!$ ways. \pause However, in
      this case, we consider the distribution of presents, i.e., we
      consider the set of presents each child gets. \pause
    \item To see how many times each distribution is counted in the $9!$
      ways, we can let children form a line and let each child permute
      his or her presents.  Each child has $3!$ choices.  Thus, one
      distribution appears $3!3!3!$ times. \pause
    \item Thus, the number of ways we can distribute presents is
      \[ 
      \frac{9!}{3!3!3!}
      \]
    \end{itemize}
  }
\end{frame}

\begin{frame}\frametitle{Another way to look at the present distribution}
  \begin{itemize}
  \item Let's look closely at a particular present distribution in the
    previous question.  Let $\{1,2,\ldots,9\}$ be the set of presents.
  \item Consider the case where A gets $\{1,3,8\}$, B gets
    $\{2,4,6\}$, and C gets $\{5,7,9\}$. \pause
  \item Another way to look at this distribution is to fix the order
    of the presents and see who gets each of the presents.  Thus, the
    previous distribution is represented in the following table:
    \begin{tabular}{|c|c|c|c|c|c|c|c|c|c|}
      Presents & 1 & 2 & 3 & 4 & 5 & 6 & 7 & 8 & 9\\ \hline
      Children & A & B & A & B & C & B & C & A & C
    \end{tabular}
  \item \pause This is essentially an anagram problem.  You can think
    of one particular way of present distribution as anagram of
    AAABBBCCC.  Thus, we reach the same solution of
    \[\frac{9!}{3!3!3!}.\]
  \end{itemize}
\end{frame}

\begin{frame}\frametitle{Distributing identical presents}
  \begin{tcolorbox}
    Now suppose that I have $9$ identical presents.  I want to give
    them to $3$ students: A, B, and C.  I want to give each student
    $3$ presents.  In how many ways can I do it?
  \end{tcolorbox}
  \begin{itemize}
  \item Note that when we state that the presents are identical, we
    mean that we do not distinguish them, i.e., the first present and
    the second present are indistinguishable.
  \end{itemize}
  \vspace{1in}
\end{frame}

\begin{frame}\frametitle{Distributing coins (1)}
  \begin{tcolorbox}
    I have $9$ indentical coins.  I want to give them to $3$ students:
    A, B, and C.  In how many ways can I do it so that each student
    gets at least one coin?
  \end{tcolorbox}

  \begin{itemize}
  \item Let's first try to organize the distribution of coins.  \pause
    We place all 9 coins in a line.  We let the first student picks
    some coin, then the second student, then the last one. \pause
  \item Since each coin is identical, we can let the first student
    picks the coin from the beginning of the line.  Then the second
    one pick the next set of coins, and so on. \pause
  \item One possible distribution is
    \[
    \underbrace{o o}_{1} \underbrace{o o o o}_{2} \underbrace{o o o}_{3}
    \]
    \pause
  \item In how many ways can we do that?
  \end{itemize}
\end{frame}

\begin{frame}\frametitle{Distributing coins (2)}
  The example below provides us with a hint on how to count.
  \[
  \underbrace{o o}_{1} \underbrace{o o o o}_{2} \underbrace{o o o}_{3}
  \]
  \pause

  Since all coins are identical, what matters are where the first
  student and the second student stop picking the coins. \pause
  I.e, the previous example can be depicted as
  \[
  o o | o o o o | o o o
  \]

  Thus, in how many ways can we do that? \pause
  
  Since there are 8 places we can mark starting points, and
  there are 2 starting points we have to place, then there are
  $\binom{8}{2}$ ways to do so. \pause

  This is a fairly surprising use of binomial coefficients.
\end{frame}

\begin{frame}\frametitle{Distributing coins (3)}
  Let's consider a general problem where we have $n$ identical coins
  to give out to $k$ students so that each student gets at least one
  coin.  In how many ways can we do that?

  \pause Since there are $n-1$ places between $n$ coins and we need to
  place $k-1$ starting points, there are $\binom{n-1}{k-1}$ ways to do
  so.

  \pause
  \begin{tcolorbox}
    There are $\binom{n-1}{k-1}$ ways to distribute $n$ identical
    coins to $k$ children so that each child get at least one coin.
  \end{tcolorbox}
\end{frame}

\begin{frame}\frametitle{Distributing coins (4)}
  \begin{tcolorbox}
    I have $9$ indentical coins.  I want to give them to $3$ students:
    A, B, and C.  In how many ways can I do it, given that some
    student may not get any coins?
  \end{tcolorbox}
  
  \vspace{1.5in}
\end{frame}

\begin{frame}\frametitle{Identities in the Triangle}
  \begin{tcolorbox}
    {\footnotesize
      \begin{tabular}{ccccccccccccccc}
        & & & & & & & 1 & & & & & & & \\
        & & & & & & 1 & & 1 & & & & & & \\
        & & & & & 1 & & 2 & & 1 & & & & & \\
        & & & & 1 & & 3 & & 3 & & 1 & & & & \\
        & & & 1 & & 4 & & 6 & & 4 & & 1 & & & \\
        & & 1 & & 5 & & 10 & & 10 & & 5 & & 1 & & \\
        & 1 & & 6 & & 15 & & 20 & & 15 & & 6 & & 1 & \\
        1 & & 7 & & 21 & & 35 & & 35 & & 21 & & 7 & & 1 \\
      \end{tabular}
    }
    \vspace{0.2in}
  \end{tcolorbox}
\end{frame}

\begin{frame}\frametitle{Odd and even subsets}
  \begin{tcolorbox}
    {\footnotesize
      \begin{tabular}{ccccccccccccccc}
        & & & & & & & 1 & & & & & & & \\
        & & & & & & 1 & & 1 & & & & & & \\
        & & & & & 1 & & 2 & & 1 & & & & & \\
        & & & & 1 & & 3 & & 3 & & 1 & & & & \\
        & & & 1 & & 4 & & 6 & & 4 & & 1 & & & \\
        & & 1 & & 5 & & 10 & & 10 & & 5 & & 1 & & \\
        & 1 & & 6 & & 15 & & 20 & & 15 & & 6 & & 1 & \\
        1 & & 7 & & 21 & & 35 & & 35 & & 21 & & 7 & & 1 \\
      \end{tabular}
    }
  \end{tcolorbox}

  Let's try to prove this identity with the Pascal's triangle
  \[
  {n\choose 0} - {n\choose 1} + {n\choose 2} +\cdots +(-1)^{n}{n\choose n} = 0.
  \]
\end{frame}

\begin{frame}\frametitle{A more formal proof}
  \begin{tcolorbox}
    \[
    {n\choose 0} - {n\choose 1} + {n\choose 2} +\cdots +(-1)^{n}{n\choose n} = 0.
    \]
  \end{tcolorbox}
  \vspace{2in}
\end{frame}

\begin{frame}\frametitle{The next experiment}
  \begin{tcolorbox}
    {\footnotesize
      \begin{tabular}{ccccccccccccccc}
        & & & & & & & 1 & & & & & & & \\
        & & & & & & 1 & & 1 & & & & & & \\
        & & & & & 1 & & 2 & & 1 & & & & & \\
        & & & & 1 & & 3 & & 3 & & 1 & & & & \\
        & & & 1 & & 4 & & 6 & & 4 & & 1 & & & \\
        & & 1 & & 5 & & 10 & & 10 & & 5 & & 1 & & \\
        & 1 & & 6 & & 15 & & 20 & & 15 & & 6 & & 1 & \\
        1 & & 7 & & 21 & & 35 & & 35 & & 21 & & 7 & & 1 \\
      \end{tabular}
    }
  \end{tcolorbox}

  Let's try to compute the sum of squares of numbers in each row.
  \begin{eqnarray*}
    1^2 &=& 1\\ \pause
    1^2 + 1^2 &=& 2 \\ \pause
    1^2 + 2^2 + 1^2 &=& 6 \\ \pause
    1^2 + 3^2 + 3^2 + 1^2 &=& 20 \\ \pause
    1^2 + 4^2 + 6^2 + 4^2 + 1^2 &=& 70 \\
  \end{eqnarray*}
\end{frame}

\begin{frame}
  \textcolor{blue}{Theorem:}
  \[
  \binom{n}{0}^2 + \binom{n}{1}^2 + \binom{n}{2}^2 + \cdots+ \binom{n}{n}^2
  = \binom{2n}{n}.
  \]
  \vspace{2.5in}
\end{frame}

\begin{frame}\frametitle{Another identity}
  \begin{tcolorbox}
    {\footnotesize
      \begin{tabular}{ccccccccccccccc}
        & & & & & & & 1 & & & & & & & \\
        & & & & & & 1 & & 1 & & & & & & \\
        & & & & & 1 & & 2 & & 1 & & & & & \\
        & & & & 1 & & 3 & & 3 & & 1 & & & & \\
        & & & 1 & & 4 & & 6 & & 4 & & 1 & & & \\
        & & 1 & & 5 & & 10 & & 10 & & 5 & & 1 & & \\
        & 1 & & 6 & & 15 & & 20 & & 15 & & 6 & & 1 & \\
        1 & & 7 & & 21 & & 35 & & 35 & & 21 & & 7 & & 1 \\
      \end{tabular}
    }
  \end{tcolorbox}
  \pause

  This suggests
  \[
  \binom{n}{0} + \binom{n+1}{1} + \binom{n+2}{2} + \cdots + \binom{n+k}{k} = \binom{n+k+1}{k}.
  \]
\end{frame}

\begin{frame}
  \textcolor{blue}{Theorem:}
  \[
  \binom{n}{0} + \binom{n+1}{1} + \binom{n+2}{2} + \cdots + \binom{n+k}{k} = \binom{n+k+1}{k}.
  \]
  \vspace{2.5in}
\end{frame}
