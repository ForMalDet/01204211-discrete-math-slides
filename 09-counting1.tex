\newcommand{\lecturetitle}[1]{
  \title{01204211 Discrete Mathematics \\ #1}
  \author{Jittat Fakcharoenphol}
  \frame{\titlepage}
}

\lecturetitle{Lecture 9: Counting 1}

\begin{frame}\frametitle{Let's count}
  \begin{tcolorbox}
    {\bf Club representatives:} You are a second year student. Your
    board game club has 40 members which are in the first year.  There
    is a big competition very soon, so the club has to find exactly 2
    representatives (from the first-year students) for the competition.
  \end{tcolorbox}
  \pause

  \begin{itemize}
  \item
    How to find these 2 representatives?  One of your friends suggests
    that to be fair to everyone, you have to look at every possible
    pair and see how the 2 members of the pair play together as a
    team.  \pause
  \item
    It might take a very long time, you think.  How many pairs are
    there?
  \end{itemize}
\end{frame}

\begin{frame}\frametitle{Club representatives (1)}
  \begin{itemize}
  \item
    To choose the member of the pair, you pick the first member and then
    pick the second member.  \pause
  \item There are 40 ways to choose the
    first member. \pause For every person you pick as the first member,
    there are exactly 39 left to pick as the second one. \pause
    Therefore, there are $40\cdot 39 = 1,560$ ways.  \pause
  \item
    Wait.. \pause This is over counting.  Picking $a$ as the first
    member and $b$ as the second member results in the same pair as
    picking $b$ first and $a$ second. \pause Thus, every pair is counted
    twice.  \pause
  \item The correct number of pairs is 780; too many possibilities to
    consider, you conclude.
  \end{itemize}
\end{frame}

\begin{frame}\frametitle{Club representatives (2)}
  \begin{itemize}
  \item Since 780 is too many, you decide to randomly choose 15 pairs
    of representatives and observe how each pair plays.
    \pause
  \item Your friend argue that 15 is too small.  Because the number of
    members is 40 and we will miss someone there.
    \pause
  \item So you ask, how many pairs one have to randomly choosing a
    pair from 40 members so that it is very likely that every member
    is picked once?
    \pause
  \item You try to calculate the number, but your friend starts
    writing a program to simulate.
  \end{itemize}
\end{frame}

\begin{frame}\frametitle{Club representatives (2)}
  \begin{itemize}
  \item Here's the table of the simulation.  For each value of number
    of random pairs, 2,000 simulations has been conducted.

    \vspace{0.1in}
    
    {\small
    \begin{tabular}{c|c}
      \hline
      Number of pairs to random & \% of choosing everyone once \\
      \hline
      20 & 0.00 \\
      30 & 0.00 \\
      40 & 0.15 \\
      50 & 2.45 \\
      60 & 12.05 \\
      80 & 51.65 \\
      100 & 78.00 \\
      120 & 91.25 \\
      140 & 97.10 \\ \hline
    \end{tabular}
    }
    \pause

  \item You end up choosing randomly 100 pairs, as it has about 80\%
    chance.  You feel so tired, but you keep wondering if you can
    calculate the number without having to write a program.
  \end{itemize}
\end{frame}

\begin{frame}\frametitle{Club representative again}
  \begin{tcolorbox}
    {\bf A team:} Another team competition is coming up.  It requires
    a team of 5 players.  In the team, each player can play either as
    Protoss, Terrans, or Zerg.  Luckily, only one team of 5 members
    volunteers to participate.
  \end{tcolorbox}

  \begin{itemize}
  \item
    To find the best team organization, you ask them to try all
    possible configurations of race choices against AI players.  How
    many games do you have to watch?  \pause
  \item
    Each member has 3 choices and this member's choice is independent
    of the other.  Therefore, there are $3\cdot 3\cdot 3\cdot 3\cdot 3
    = 243$ possible ways. \pause
  \item
    You are still tired from watching 100 pairs of players.  So you
    change your mind and ask them to try only configurations that
    contain all the three races.  How many are there?
  \end{itemize}
\end{frame}
