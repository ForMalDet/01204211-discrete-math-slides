\newcommand{\lecturetitle}[1]{
  \title{01204211 Discrete Mathematics \\ #1}
  \author{Jittat Fakcharoenphol}
  \frame{\titlepage}
}

\lecturetitle{Lecture 16: Primality testing (1)} 

\begin{frame}\frametitle{Integers}
  This is the second major area that we shall study in this course:
  integers and their properties.  This area is called {\bf number
    theory.}  We will see many properties of integers and their
  applications that include data encoding techniques and data
  encryption.

  Most techniques depend on having large primes.  So, we start this
  area by focus on the primality testing algorithm.  This is also the
  first algorithm that we considered in this class as well.
\end{frame}

\begin{frame}\frametitle{Brute-force division}
  \begin{tcolorbox}
    {\bf FUNCTION} CheckPrime2($n$)\\
    1. $k\leftarrow 2$\\
    2. WHILE $k\leq\sqrt{n}$ DO\\
    3. \ \ IF $k$ divides $n$ THEN\\
    4. \ \ \ \ RETURN {\bf false}\\
    5. \ \ ENDIF\\
    6. \ \ $k\leftarrow k+1$\\
    7. ENDWHILE\\
    8. RETURN {\bf true}
  \end{tcolorbox}

  The WHILE loop in CheckPrime2 runs for at most $\sqrt{n}$ times.  This
  improves over the original algorithm that uses roughly $n$ rounds.
  While this is a big improvement, it is usually not good enough when
  we consider a typical usage of the algorithm that we need, where we
  need to check if a 1000-digit number is a prime.
\end{frame}

\begin{frame}\frametitle{Running time analysis: the $O$-notation}
  \begin{itemize}                  
  \item We shall study the running time of various algorithms in this
    section.  Since we will not be very precise, keeping tracks of all
    details, we will use the $O$-notation when we talk about the
    running time.
  \item
    We will informally use the notation.  When we say that function
    $f(n)$ is $O(g(n))$, we means that the growth of $f(n)$ is at most
    that of $g(n)$.  This means that the largest terms in $f(n)$ is
    not larger than that of $g(n)$.
  \item
    {\bf Examples:}
    \begin{itemize}
    \item $n^2 + 100n = O(n^2)$
    \item $\sqrt{n} + n = O(n)$
    \item $2^n + 10000 + 100000n^{10} = O(2^n)$
    \item $5\cdot n^3 + n^2 = O(n^3)$
    \item $10n^2 \times 7n + 12n^3\times 75n^2  = O(n^3) + O(n^5) = O(n^5)$
    \end{itemize}
  \item Note that it is also true that $n^2 = O(n^3)$.
  \end{itemize}
\end{frame}

\begin{frame}\frametitle{Polynomial running times}
  Let's discuss CheckPrime2's running time.

  We usually think about the running time as a function on the size of
  the input.  When we want to sort $n$ numbers, the size of the input
  is $n$.  However, when dealing with big numbers (i.e., those much
  larger than directly manipulatable in the CPU), you cannot
  manipulate them in constant time.  In this case, we usually count
  the number of bits as the size of the input.

  Since $n$ is the value of the input, to keep this integer in
  computer memory, you need at least $\log_2 n$ bits\footnote{We shall
    use logarithm base 2 in this part of the course.}.  This will be
  the size of the input that we shall consider.

  \vspace{0.1in}
  Let $m=O\log n$ be the number of bits of $n$.  The algorithm that
  runs in time $O(\sqrt{n})$, actually runs in time
  $O(\sqrt{2^m})=O(2^{m/2})$, an exponential running time.  This means
  that the algorithm does not scale very well, as the size of the
  input increases.
  
  In this case, we want a more efficient algorithm, i.e., it runs in
  time in the polynomial of $m$.
\end{frame}

\begin{frame}\frametitle{Running time: integer addition}
  
\end{frame}

\begin{frame}\frametitle{Running time: integer multiplication}
\end{frame}

\begin{frame}
  This lecture covers
  \begin{itemize}
  \item basic definitions related to division and modulo operation
  \item prime factorization
  \item a fundamental theorem stating a fact related to prime numbers
    called the Fermat's Little Theorem
  \end{itemize}
\end{frame}

\begin{frame}\frametitle{Definitions: divisibilty}
  Let's start with basic definitions.
  \begin{itemize}
  \item We say that ``$a$ divides $b$'', ``$b$ is divisible by $a$'',
    or ``$b$ is a multiple of $a$'' if there exists an integer $k$
    such that $b = ka$.  In this case, we write \[a|b.\]
  \item If it's not the case, we write $a\not| b$.
  \item When $a$ does not divides $b$, there is a remainder.  We say
    that $r$ is a remainder of dividing $b$ by $a$ if $0\leq r<a$ and
    there exists integer $k$ such that $b = ka + r$.  We also write
    \[ r = a\mod b.\]
    \begin{itemize}
    \item $10\mod 3 = 1$, $10\mod 2 = 0$, $10\mod 15=10$
    \item $-10\mod 3 = 2$, $-10\mod 15=5$
    \end{itemize}
  \end{itemize}
\end{frame}

\begin{frame}\frametitle{The modulo operation}
  \begin{tcolorbox}
    For integers $a$ and $b$ and positive integer $q$, we have
    \begin{itemize}
    \item $(a+b)\;\bmod\;q = ((a\;\bmod\;q) + (b\;\bmod\;q))\;\bmod\; q$
    \item $(a-b)\;\bmod\;q = ((a\;\bmod\;q) - (b\;\bmod\;q))\;\bmod\; q$
    \item $(ab)\;\bmod\;q = ((a\;\bmod\;q) \times (b\;\bmod\;q))\;\bmod\; q$
    \end{itemize}
  \end{tcolorbox}

  Examples:
  \begin{itemize}
    \item $(14 + 7)\;\bmod\; 5 = ((14\;\bmod\;5) + (7\;\bmod\;5))\;\bmod\;5 = (4+2)\;\bmod\;5 = 1$
    \item $(14\cdot 7\cdot 13\cdot 19)\;\bmod\; 5 = ((14\;\bmod\;5)\cdot(7\;\bmod\;5)\cdot(13\;\bmod\;5)\cdot(19\;\bmod\;5))\;\bmod\;5 = (4\cdot 2\cdot 3\cdot 4)\;\bmod\;5 = 96\;\bmod\;5=1$
  \end{itemize}

  {\small These facts are really helpful when you try to compute
    $x\;\bmod\; y$ when $x$ is very large compared to $y$ and $x$ is a
    result of many operations of small numbers.  In this case, we can
    keep moduloing intermediate results to keep them under $y$.
  
    You will prove these properties in your homework.
  }
\end{frame}

\begin{frame}\frametitle{Definitions: primes}
  \begin{tcolorbox}
    An integer $p$ is a {\bf prime} if $p>1$ and $p$ has only 4
    factors: $1,-1,p,$ and $-p$.  If a number larger than $1$ is not a
    prime, we say that it is a {\bf composite}.
  \end{tcolorbox}

  Prime numbers are very fascinating.  There are many facts that have
  proved about them.  E.g., we looked at Euclid's proof that there are
  infinitely many primes.  Here's another one by Euclid:

  {\small
    \begin{tcolorbox}
      \textcolor{blue}{Theorem:} For any positive integer $n$, there
      are $n$ consecutive composites.
      
      \textcolor{blue}{Proof:} Let $m=n+1$.  Consider
      \[
      (m!+2),(m!+3),\ldots,(m!+m).
      \]
      Note that these $m-1=n$ numbers are composite because for any
      $1\leq i\leq m$, $i|m!$, $i|i$, and thus,
      $i|(m!+i)$. $\blacksquare$
    \end{tcolorbox}
  }
\end{frame}

\begin{frame}\frametitle{Prime factorization}
  It is known since the Greeks that if you have a composite $n$, you
  can factor it as a product of prime numbers.  For example, you can
  write
  \[ 140 = 2\times 2\times 5\times 7. \]
\end{frame}

\begin{frame}
  \textcolor{blue}{Theorem:} A prime factorization of a positive
  number larger than 1 is unique.
\end{frame}

\begin{frame}
  \textcolor{blue}{Lemma:} If $p$ is a prime for any integer $1\leq
  k<p$, we have that
  \[
  p\Big|\binom{p}{k}.
  \]
\end{frame}

\begin{frame}\frametitle{Fermat's Little Theorem}
  \begin{tcolorbox}
    \textcolor{blue}{Theorem:} If $p$ is a prime and for any integer
    $a$, we have that $p|a^p-a$.
  \end{tcolorbox}
\end{frame}
