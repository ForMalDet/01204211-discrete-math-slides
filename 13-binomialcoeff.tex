\newcommand{\lecturetitle}[1]{
  \title{01204211 Discrete Mathematics \\ #1}
  \author{Jittat Fakcharoenphol}
  \frame{\titlepage}
}

\lecturetitle{Lecture 13: Binomial Coefficients} 

\begin{frame}\frametitle{The binomial coefficients}
  There is a reason why the term $\binom{n}{k}$ is called the binomial
  coefficients.  In this lecture, we will discuss
  \begin{itemize}
  \item the Pascal's triangle, 
  \item the binomial theorem, and
  \item advanced counting with binomial coefficients.
  \end{itemize}
\end{frame}

\begin{frame}\frametitle{The equation}
  Last time we proved that, for $n,k>0$,
  \[\binom{n}{k} = \binom{n-1}{k-1} + \binom{n-1}{k}.\]
  \pause

  While we can prove this equation algebraically using definitions of
  binomial coefficients, proving the fact by describing the process of
  choosing $k$-subsets reveals interesting insights.  This equation
  also hints us how to compute the value of $\binom{n}{k}$ using
  values of $\binom{n}{\cdot}$'s.

  \pause
  So, let's try to do it.
\end{frame}

\begin{frame}\frametitle{The table}
  We shall use the fact that $\binom{n}{0}=1$ and $\binom{n}{k} =
  \binom{n-1}{k-1} + \binom{n-1}{k}$ to fill in the following table.

  \begin{tabular}{|r|c|c|c|c|c|c|c|}
    \hline
    $n$ & 0 & 1 & 2 & 3 & 4 & 5 & 6 \\ 
    \hline
    $0$ & 1 &&&&&&\\
    \hline
    $1$ & 1 & 1 &&&&&\\
    \hline
    $2$ & 1 & \pause 2 & 1 &&&&\\
    \hline
    \pause
    $3$ & 1 & \pause 3 & 3 & 1 &&&\\
    \hline
    \pause
    $4$ & 1 & \pause 4 & 6 & 4 & 1 &&\\
    \hline
    \pause
    $5$ & 1 & \pause 5 & 10 & 10 & 5 & 1 &\\
    \hline
    \pause
    $6$ & 1 & \pause 6 & 15 & 20 & 15 & 6 & 1 \\
    \hline
  \end{tabular}
  
  \vspace{0.1in}

  \pause You can note that the table is left-right symmetric.  This is
  true because of the fact that $\binom{n}{k} = \binom{n}{n-k}$.
\end{frame}

\begin{frame}\frametitle{The Triangle}
  If we move the numbers in the table slightly to the right, the table
  becomes the Pascal's triangle.
  \pause

  \vspace{0.1in}

  \begin{tcolorbox}
  \begin{tabular}{ccccccccccccc}
    & & & & & & 1 & & & & & & \\
    & & & & & 1 & & 1 & & & & & \\
    & & & & 1 & & 2 & & 1 & & & & \\
    & & & 1 & & 3 & & 3 & & 1 & & & \\
    & & 1 & & 4 & & 6 & & 4 & & 1 & & \\
    & 1 & & 5 & & 10 & & 10 & & 5 & & 1 & \\
    1 & & 6 & & 15 & & 20 & & 15 & & 6 & & 1 \\
    & $\vdots$ & & $\vdots$ & & & & & & & & $\vdots$ & \\
  \end{tabular}
  \end{tcolorbox}

  \vspace{0.1in}
  
  The table and the binomial coefficients have many other interesting
  properties.
\end{frame}

\begin{frame}\frametitle{Polynomial expansions}
  Let's start by looking at polynomial of the form $(x+y)^n$.  Let's
  start with small values of $n$:
  \begin{itemize}
  \item $(x+y)^1=x+y$
  \item $(x+y)^2 = \pause x^2 + 2\cdot xy + y^2$\\
  \item \pause $(x+y)^3 = \pause x^3 + 3\cdot x^2y + 3\cdot xy^2 + y^3$\\
  \item \pause $(x+y)^4 = \pause x^4 + 4\cdot x^3y + 6\cdot x^2y^2 + 4\cdot xy^3 + y^4$.
  \end{itemize}
  
  \vspace{0.1in}
  Let's focus on the coefficient of each term.  You may notice that
  terms $x^n$ and $y^n$ always have 1 as their coefficients.  {\em Why
    is that?} \pause

  Let's look further at the coefficients of terms $x^{n-1}y$.  Do you
  see any pattern in their coefficients?  {\em Can you explain why?}
\end{frame}
