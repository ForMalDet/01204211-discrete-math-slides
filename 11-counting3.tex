\newcommand{\lecturetitle}[1]{
  \title{01204211 Discrete Mathematics \\ #1}
  \author{Jittat Fakcharoenphol}
  \frame{\titlepage}
}

\lecturetitle{Lecture 11: Counting 3}

\begin{frame}\frametitle{Quick recap}
  We have proved many useful facts.
  \begin{itemize}
  \item The number of subsets of a set with $n$ elements is
    $2^n$. \pause In fact, we know 3 proofs of this fact:
    \begin{itemize}
    \item We count the number of ways one can choose a subset.
    \item We provide a bijection between subsets and binary strings.
    \item We prove the fact by induction.
    \end{itemize}
    \pause
  \item For a set with $n$ elements, the number of its permutations is
    $n!$.
  \end{itemize}
\end{frame}

\begin{frame}\frametitle{This lecture's goals}
  \begin{itemize}
  \item Consider set $\{1,2,3,4,5\}$.  How many subsets with $10$
    elements does this set have? \pause
  \item There are $10$ subsets with $3$ elements: $\{1,2,3\}$,
    $\{1,2,4\}$, $\{1,2,5\}$, $\{1,3,4\}$, $\{1,3,5\}$, $\{1,4,5\}$,
    $\{2,3,4\}$, $\{2,4,5\}$, $\{2,4,5\}$, $\{3,4,5\}$.
  \item In this lecture, we shall find out how to count these subsets. \pause
  \end{itemize}

  \begin{tcolorbox}
    {\bf\textcolor{blue}{Abbreviations:}} We shall call a set with $n$ elements as an
    {\bf $n$-set}.  We shall call a subset with $k$ elements as a {\bf
      $k$-subset}.
  \end{tcolorbox}

  \pause

  \begin{itemize}
  \item We will also discuss the inclusion-exclusion priciples.
  \end{itemize}
\end{frame}

\begin{frame}\frametitle{Ordered subsets}
  In general, elements in a given set is unordered.  I.e., sets
  $\{1,2,3\}$ and $\{3,1,2\}$ are the same set. \pause

  However, sometimes, it is useful to treat sets as ordered. \pause

  For example, for set $\{1,2,3\}$, there are $6$ ordered subsets with
  $2$ elements: $\{1,2\}$, $\{1,3\}$, $\{2,1\}$, $\{2,3\}$, $\{3,1\}$,
  $\{3,2\}$.
\end{frame}

\begin{frame}\frametitle{Example: runners}
  \begin{tcolorbox}
    {\bf Question:} There are 10 runners for a given competition.
    There are 3 awards: 1st price, 2nd price and 3rd price.  In how
    many possible ways these 3 awards can be given?  (No runner can
    get more than one award.)
  \end{tcolorbox}
  \pause

  We can use the argument we used to derive the number of permutations
  here.  We consider the process for selecting the winners. \pause
  
  \begin{itemize}
  \item First, we pick the 1st price winner: there are 10 choices. \pause
  \item For any 1st price winner, there are 9 choices to choose the
    2nd price winner. \pause
  \item For any 1st and 2nd price winners, there are 8 choices for the
    3rd winner. \pause
  \item Therefore, we conclude that the number of ways is $10\cdot
    9\cdot 8$.
  \end{itemize}
\end{frame}

\begin{frame}\frametitle{Example: runners (another look)}
  We can arrive at the same answer by a different way of counting.
  
  \begin{itemize}
  \item Let's count all possible running results: there are $10!$
    results.  (I.e., each running result is a permutation.) \pause
    \begin{itemize}
    \item $10!$ is too many for our answer. Why? \pause
    \end{itemize}
  \item For a particular selection of 3 top winners, how many possible
    running results have exactly these 3 top winners? \pause
    \begin{itemize}
    \item The number of running results is the number of permutation
      of the other 7 non-winning runners; thus, there are $7!$ of
      them.
    \end{itemize}
  \item We can think of a process of choosing a permutation as having
    two big steps: (1) pick 3 top winners, then (2) pick the rest of
    runners.  This provide a different way to count the number of
    permutations.
    \pause
  \item Let $X$ be the set of ordered subsets with $3$ elements of an
    $10$-set.  We then have $|X|\times 7! = 10!$, because they count
    the same objects. \pause  Solving this yields
    \[ |X| = \frac{10!}{7!} = 10\cdot 9\cdot 8.\]
  \end{itemize}
\end{frame}

\begin{frame}\frametitle{General answers: numbers of ordered subsets}
  Using the same arguments (either one), we have this theorem.

  \begin{theorem}
    The number of ordered subsets with $k$ elements of an $n$-set is
    \[
    n\cdot(n-1)\cdots(n - k + 1) = \frac{n!}{(n-k)!}.
    \]
  \end{theorem}
\end{frame}

\begin{frame}\frametitle{How big is $100!$ \ ?}
  \begin{itemize}
  \item
    With computers, we may be able to answer the exact long number.
    But mathematicians usually enjoy a ``quick'' estimate just to have
    a rough idea on how things are.

  \item
    How can we start? \pause When we want to get an estimate, we
    usually start by finding an {\bf upper bound} and a {\bf lower
      bound} for the quantity. \pause As the names suggest, the upper
    bound for $x$ is a quantity that is not smaller than $x$, and the
    lower bound for $x$ is a quantity that is not larger than $x$
    (maybe under some condition). \pause

  \item
    Let's think about $n!$.  \pause
    \begin{itemize}
    \item The first lower bound that comes to mind for $n!$ is $1^n =
      1$. \pause
    \item Can we get a better lower bound?  (Here, better lower bounds
      should be closer to the actual value.) \pause How about $2^n$?
      Is it a lower bound? \pause How about $3^n$ or $5^n$? Are they
      lower bounds of $n!$?
    \end{itemize}
  \end{itemize}
\end{frame}

\begin{frame}\frametitle{Bounds for $n!$}
  Recall that $n!=1\cdot 2\cdot 3\cdots n$.  Since all its factor,
  except the first one is at least 2, we have that
  \[ 2^{n-1}\leq n!. \] \pause
  Similarly, since all factors of $n!$ is at most $n$, we have that
  \[ n! \leq n^n.\] \pause
  A slightly better upper bound is $n^{n-1}$ because we can, again,
  ignore $1$.
  \pause

  \vspace{0.1in}
  Are they any good? \pause

  \begin{tabular}{c|c|c|c}
    $n$ & $2^{n-1}$ & $n!$ & $n^{n-1}$ \\ \hline
    $1$ & $1$ & $1$ & $1$ \\
    $2$ & $2$ & $2$ & $2$ \\
    $3$ & $4$ & $6$ & $9$ \\
    $4$ & $8$ & $24$ & $64$ \\
    $10$ & $512$ & $3,628,800$ & $1,000,000,000$
  \end{tabular}
\end{frame}


\begin{frame}\frametitle{Stering's formula}
  A better estimate for $n!$ exists.

  \begin{theorem}[Stering's formula]
    $n!\sim \left(\frac{n}{e}\right)^n\sqrt{2\pi n}$.
  \end{theorem}

  \vspace{0.2in}
  When we write $a(n)\sim b(n)$, we mean that
  $\frac{a(n)}{b(n)}\rightarrow 1$ as $n\rightarrow\infty$.
  \pause

  With Stering's formula, We can use a calculator to estimate the
  number of digits for $100!$.  \pause The estimate for $100!$ is
  \[ (100/e)^{100}\cdot\sqrt{200\pi}\]
  Thus, the number of digits is its logarithm, in base 10, i.e.,
  \[\log\left((100/e)^{100}\cdot\sqrt{200\pi}\right) = 100\log(100/e) + \log(200\pi) \approx 157.9696.\]
  \pause
  Note that the correct answer is 158 digits.
\end{frame}

\begin{frame}\frametitle{The number of subsets}
  \begin{tcolorbox}
    \textcolor{blue}{Theorem:}
    The number of $k$-subsets of an $n$-set is
    \[\frac{n\cdot(n-1)\cdot(n-2)\cdots(n-k+1)}{k!}=\frac{n!}{(n-k)!k!}.\]
  \end{tcolorbox}
  
  \begin{proof} {\small
    \pause Consider the following process for choosing an ordered
    subsets with $k$ elements of an $n$-set.  \pause First, we choose
    a $k$-subset, then we permute it.  Let $B$ be the number of
    $k$-subsets.  For each subset that we choose in the first step,
    the second step has $k!$ choices.  \pause Therefore, we can choose
    an ordered subset in $B\cdot k!$ possible ways.  \pause From the
    previous discussion, we know that
    \[
    B\cdot k! = n\cdot(n-1)\cdots(n-k+1).
    \]
    Therefore, the number of $k$-subsets is
    \[\frac{n\cdot(n-1)\cdot(n-2)\cdots(n-k+1)}{k!}=\frac{n!}{(n-k)!k!},\]
    as required.
    }
  \end{proof}
\end{frame}

\begin{frame}\frametitle{Binomial coefficients}
  The number of $k$-subsets of an $n$-set is very useful.  Hence,
  there is a notation for it, i.e.,
  \[
  {n\choose k} = \frac{n!}{(n-k)!k!},
  \]
  (which reads ``$n$ choose $k$'').  These numbers are called {\bf
    binomial coefficients}. \pause

  \vspace{0.1in}
  
  Note that
  \begin{itemize}
  \item ${n\choose n}=1$ (why?), \pause
  \item ${n\choose 0}=1$ (why?), \pause and,
  \item when $k>n$, ${n\choose k}=0$.
  \end{itemize}
\end{frame}

\begin{frame}\frametitle{Properties (1)}
  \textcolor{blue}{Theorem:} $${n\choose k}={n\choose n-k}.$$

  \vspace{2.5in}
\end{frame}

\begin{frame}\frametitle{Properties (2)}
  \textcolor{blue}{Theorem:} When $n,k>0$, then
  $${n\choose k}={n-1\choose k-1} + {n-1\choose k}.$$

  \vspace{2.5in}
\end{frame}

\begin{frame}\frametitle{Properties (3)}
  \textcolor{blue}{Theorem:} When $n,k>0$, then
  $${n\choose 0}+{n\choose 1}+{n\choose 2}+\cdots+{n\choose n}=2^n.$$

  \vspace{2.5in}
\end{frame}

\begin{frame}\frametitle{Quick questions (1)}
  \begin{tcolorbox}
    There are 40 students in the classroom.  There are 35 students who
    like Naruto, 10 students who like Bleach, and 7 students who like
    both of them.  How many students in this classroom who do not like
    either Bleach or Naruto?
  \end{tcolorbox}

  \vspace{2in}
\end{frame}

\begin{frame}\frametitle{Quick questions (2)}
  \begin{tcolorbox}
    There are 35 students in the classroom.  There are 25 students who
    like Naruto, 15 students who like Bleach, 12 students who like One
    Piece.  There are 10 students who like both Naruto and Bleach, 7
    students who like both Bleach and One Piece, and 9 students who
    like both Naruto and One Piece.  There are 5 students who like all
    of them.

    How many students in this classroom who do not like any of Bleach,
    Naruto, or One Piece?
  \end{tcolorbox}

  \vspace{1.2in}
\end{frame}

\begin{frame}\frametitle{Is this correct?}
  The answer from the previous quick question is
  \[ 35 - (25 + 15 + 12 - 10 - 7 - 9 + 5) = 4.\]

  Is this correct?  Why?

  \pause

  \vspace{0.2in}

  Let's try to argue that this answer is, in fact, correct and try to
  find general answers to this kind of counting questions.
\end{frame}

\begin{frame}\frametitle{Let's look at an individual student (1)}
  {\small
  \begin{tabular}{c|c||c|c|c|c|c|c|c|c||c}
    & &  & N & B & O & NB & BO & NO & NBO & \\
    \hline
    & & $35$ & $-25$ & $-15$ & $-12$ & $+10$ & $+7$ & $+9$ & $-5$ & $4$ \\
    \hline
    Alfred & N,O & \pause * & * & & * & & & * & & \\
    Bobby & B & \pause * & & * & & & & & & \\
    Cathy & B,O & \pause * & & * & * & & * & & & \\
    Dave & N,B,O & \pause * & * & * & * & * & * & * & * & \\
    Eddy & - & \pause * & &  & & & & & & \\
    $\vdots$ & $\vdots$ & & & & & & & & & \\
  \end{tabular}
  }
\end{frame}

\begin{frame}\frametitle{Let's look at an individual student (2)}
  {\small
  \begin{tabular}{c|c||c|c|c|c|c|c|c|c||c}
    & &  & N & B & O & NB & BO & NO & NBO & \\
    \hline
    & & $35$ & $-25$ & $-15$ & $-12$ & $+10$ & $+7$ & $+9$ & $-5$ & $4$ \\
    \hline
    Alfred & N,O & 1 & -1 & & -1 & & & +1 & & 0 \\
    Bobby & B & 1 & & -1 & & & & & & 0 \\
    Cathy & B,O & 1 & & -1 & -1 & & +1 & & & 0 \\
    Dave & N,B,O & 1 & -1 & -1 & -1 & +1 & +1 & +1 & -1 & 0 \\
    Eddy & - & 1 & &  & & & & & & 1 \\
    $\vdots$ & $\vdots$ & & & & & & & & & \\
  \end{tabular}
  }
\end{frame}

\begin{frame}\frametitle{Let's see how each one is counted}
  Alfred (N,O): \pause $$1 - {2\choose 1} + {2\choose 2} = \pause 1 - 2 + 1 = 0$$

  Bobby (B): \pause $$1 - {1\choose 1} = \pause 1 - 1 = 0$$

  Dave (N,B,O): \pause $$1 - {3\choose 1} + {3\choose 2} - {3\choose 3} = \pause 1 - 3 + 3 - 1 = 0$$

  \pause

  Do you see any patterns here?
  \pause
  How about $$1 - {5\choose 1} + {5\choose 2} - {5\choose 3} + {5\choose 4} - {5\choose 5}\ \ ?$$
\end{frame}

\begin{frame}\frametitle{Underlying structures}
  Let's write $1$ as ${5\choose 0}$.  Also, let's separate plus terms
  and minus terms:

  $${5\choose 0} + {5\choose 2} + {5\choose 4}
  \ \ \ \ \ \heartsuit \ \ \ \ \
  {5\choose 1} + {5\choose 3} + {5\choose 5}$$

  \pause
  \vspace{0.1in}
  Note that the left terms are the number of even subsets and the
  right terms are the number of odd subsets.  Do you recall one of the
  homework? \pause We have proved this:

  \begin{tcolorbox}
    {\bf Theorem:} The number of even subsets is equal to the number
    of odd subsets.
  \end{tcolorbox}

  This theorem also shows that our calculation technique is correct.
  This technique is usually called the {\bf Inclusion-Exclusion
    principle}.
\end{frame}
