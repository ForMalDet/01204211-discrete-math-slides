\newcommand{\lecturetitle}[1]{
  \title{01204211 Discrete Mathematics \\ #1}
  \author{Jittat Fakcharoenphol}
  \frame{\titlepage}
}

\lecturetitle{Lecture 4: Proof techniques 1}

\begin{frame}\frametitle{Proof techniques}
  Using inference rules, we can prove facts in propositional logic.
  However, in many cases, we want to prove wider range of mathematical
  facts.  Inference rules play crucial parts in providing high-level
  structures for our proofs. \pause

  In this lecture, we will focus on two general proof techniques that
  originate from two simple inference rules.
  \begin{itemize}
  \item Direct proofs
  \item Indirect proofs
  \end{itemize}
\end{frame}

\begin{frame}\frametitle{Terminologies}
  These are terminologies used when showing mathematical facts.
  \begin{itemize}
  \item A {\bf theorem} is a statement that can be argued to be true.
  \item A {\bf proof} is the sequence of statements forming that
    mathematical argument.
    \pause
  \item An {\bf axiom} is a statement that is assumed to be true.
    (Note that we do not prove an axiom; therefore, the validity of a
    theorem proved using an axiom relies of the validity of the
    axiom.)
    \pause
  \item To prove a theorem, we may prove many simple lemmas to make
    our argument.  A {\bf lemma}, in this sense, is a smaller theorem
    (or a supportive one).
    \pause
  \item A {\bf corollary} is a theorem which is a ``fairly'' direct
    result of other theorems.
    \pause
  \item A {\bf conjecture} is a statement which we do not know if it
    is true or false.
  \end{itemize}
\end{frame}

\begin{frame}\frametitle{Fermat's Last Theorem}
  \begin{tcolorbox}
    {\bf Theorem:} No three positive integers $a$, $b$, and $c$ can satisfy the equation $a^n+b^n=c^n$ when $n>2$.
  \end{tcolorbox}

  This theorem has been conjectured by Pierre de Fermat in 1637.  It
  remained a conjecture until Andrew Wiles proved it in 1994.
\end{frame}

\begin{frame}\frametitle{Goldbach's conjecture}
  \begin{tcolorbox}
    {\bf Conjecture:} Every even integer greater than $2$ can be
    expressed as the sum of two primes.
  \end{tcolorbox}

  In 1742, Christian Goldbach proposed this cojecture to Leonhard
  Euler.  It remains unsolved.
\end{frame}

\begin{frame}\frametitle{Euclid's axioms}
  Euclidean geometry is defined by the following 5 postulates
  (axioms).
  \begin{enumerate}
  \item A straight line segment can be drawn joining any two points.
  \item Any straight line segment can be extended indefinitely in a straight line.
  \item Given any straight line segment, a circle can be drawn having the segment as radius and one endpoint as center.
  \item All right angles are congruent.
  \item (The parallel postulate) If two lines are drawn which
    intersect a third in such a way that the sum of the inner angles
    on one side is less than two right angles, then the two lines
    inevitably must intersect each other on that side if extended far
    enough.
  \end{enumerate}

  {\bf References:} Weisstein, Eric W. "Euclid's Postulates." From
  MathWorld--A Wolfram Web
  Resource. http://mathworld.wolfram.com/EuclidsPostulates.html
\end{frame}

\begin{frame}\frametitle{The triangle postulate}
  The following statement is called the triangle postulate.
  \begin{tcolorbox}
    The sum of the angles in every triangle is $180^{o}$.
  \end{tcolorbox}

  The only way to prove this in Euclidean geometry is to use the
  parallel postulate.  (Exercise: try to prove it.)\pause 

  Is this statement always true everywhere in the world (or in the
  universe)? \pause

  There are other geometries where Euclid's 5\textsuperscript{th}
  postulate is not true; then the triagle postulate may not be true in
  those cases.

  Can you imagine one?
\end{frame}

\begin{frame}\frametitle{Direct proofs}
  When we want to prove a theorem of the form $P\Rightarrow Q$, we can
  assume that $P$ is true, then use this to argue that $Q$ has to be
  true as well.

  \begin{tcolorbox}[title=Direct proofs]
    \begin{theorem}
      $P\Rightarrow Q$.
    \end{theorem}
    \begin{proof}
      Assume $P$.
      
      ... (then show that $Q$ follows from $P$)
    \end{proof}
  \end{tcolorbox}
\end{frame}

\begin{frame}\frametitle{Example 1}
  \begin{theorem}
    If $x$ is an even number, then $x^2$ is an even number.
  \end{theorem} \pause
  \begin{proof}
    Assume that $x$ is an even number. \pause

    By definition, there exists an integer $k$ such that
    $x=2k$. \pause This implies that $x^2 = (2k)^2 = 4k^2$.  \pause
    Since $k$ is an integer, $2k^2$ is also an integer.  Hence we can
    write $x^2 = 2\cdot (2k^2)$ where $2k^2$ is an integer; this means
    that $x^2$ is even.
  \end{proof}
\end{frame}

\begin{frame}\frametitle{Example 1: disected}
  \begin{theorem}
    ($\forall x$) If $x$ is an even number, then $x^2$ is an even
    number.
  \end{theorem} \pause
  \begin{proof}
    {\small
      \begin{itemize}
      \item Assume $P(x)$ where $P(x)$ = ``$x$ is an even
        number''. \pause
      \item By definition, $P(x)\Rightarrow R(x)$ where $R(x)$ =``there
        exists an integer $k$ such that $x=2k$.'' \pause
      \item $R(x)\Rightarrow S(x)$, where $S(x)$ = ``there exists an
        integer $k$ such that $x^2 = (2k)^2 = 4k^2$.''  \pause
      \item By elementary algebra, we know that $U$ is true, where $U$ =
        ``for all integer $k$, $2k^2$ is an integer.''  \pause
      \item $S(x)\wedge U\Rightarrow V(x)$, where $V$ = ``there exists
        an integer $k$ such that $x^2 = 2\cdot (2k^2)$ where $2k^2$ is
        an integer.'' \pause
      \item By definition, $V(x)\Rightarrow Q(x)$, where $Q(x)$ = ``$x^2$
        is even''.
      \end{itemize}
    }
  \end{proof}
\end{frame}

\begin{frame}\frametitle{Example 1: be careful}
  When we prove a statement with universal quantifiers like:

  \begin{tcolorbox}
    ($\forall x$) If $x$ is an even number, then $x^2$ is an even
    number
  \end{tcolorbox}

  we have to be {\em extremely} careful not to assume anything about
  $x$ except those state explicitly in the assumption.
\end{frame}

\begin{frame}\frametitle{Practice: Back to our subgoal}
  Can you use direct proofs to show the following theorem?

  \begin{theorem}
    For any positive number $n$ and $a$ such that $a > \sqrt{n}$, then
    $n/a\leq\sqrt{n}$.
  \end{theorem} \pause

  \begin{proof}
    Assume that $a > \sqrt{n}$. \pause
    Since
    \[ n = n, \]
    by dividing the left side by $a$ and the right side by $\sqrt{n}$,
    we get that
    \[ \frac{n}{a} < \frac{n}{\sqrt{n}}, \]
    because both $a$ and $\sqrt{n}$ are positive.  Hence, $n/a <
    \sqrt{n}$ as required.
  \end{proof}
\end{frame}

\begin{frame}\frametitle{Proof by contraposition}
  When we want to prove a theorem of the form $P\Rightarrow Q$, we can
  assume that $Q$ is false, then use this to argue that $P$ has to be
  false as well.

  \begin{tcolorbox}[title=Proof by contraposition]
    \begin{theorem}
      $P\Rightarrow Q$.
    \end{theorem}
    \begin{proof}
      Assume $\neg Q$.
      
      ... (then show that $\neg P$ follows from $\neg Q$)
    \end{proof}
  \end{tcolorbox}
\end{frame}

\begin{frame}\frametitle{Practice}
  \begin{theorem}
    If $x^2$ is an even number, then $x$ is an even number, 
  \end{theorem} \pause

  Before we try to prove by contraposition, let's try to use direct
  proof to show this theorem. \pause

  \begin{proof}
    Assume that $x^2$ is an even number...

    \vspace{1.2in} \pause

    ... doesn't seem to go very well.
  \end{proof}
\end{frame}

\begin{frame}\frametitle{Practice}
  \begin{theorem}
    If $x^2$ is an even number, then $x$ is an even number, 
  \end{theorem} \pause
  \begin{proof}
    We will prove by contraposition.  
    Assume that $x$ is not an even number.

    \vspace{1.5in}
  \end{proof}
\end{frame}

\begin{frame}\frametitle{An incorrect proof is not a proof (1)}
  \begin{theorem}
    For any numbers $x$ and $y$, $x = y$.
  \end{theorem}
  \pause

  \begin{proof}
    Assume that
    \[ x = y. \] \pause
    Multiplying both terms by 0, we get that
    \[ 0\cdot x = 0\cdot y, \] \pause
    and this implies
    \[ 0 = 0, \]
    which is clearly true.
  \end{proof}
\end{frame}
