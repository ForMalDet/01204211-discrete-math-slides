\documentclass{beamer}
\usefonttheme[onlymath]{serif}
\begin{document}

\newcommand{\lecturetitle}[1]{
  \title{01204211 Discrete Mathematics \\ #1}
  \author{Jittat Fakcharoenphol}
  \frame{\titlepage}
}

\lecturetitle{Lecture 1: Introduction}

\begin{frame}\frametitle{The goals of this course}
  There are two goals:
  \begin{itemize}
  \item To learn how to make mathematical arguments.
    \pause
  \item To learn various fundamental mathematical concepts that are
    very useful in computer science.
  \end{itemize}
\end{frame}

\begin{frame}\frametitle{What is mathematics?}
  \pause
  Ah... that's a philosopical question. \\
  \pause
  IMHO, mathematics is a mean to communicate {\em precise} ideas.
\end{frame}

\begin{frame}\frametitle{It's like learning a new language}
\end{frame}

%%%%%%%%%%%% why

\begin{frame}[fragile]\frametitle{Why should we learn how to prove? (1)}
  \pause
  Look at this program.

\begin{verbatim}
    if a > b:
        return a
    else:
        return b
\end{verbatim}

  The author claims that this program takes two variables $a$ and $b$
  and returns the larger one.
  \pause
  
  {\em Do you believe the author of the code? \pause Why?}
\end{frame}

\begin{frame}[fragile]\frametitle{Finding the maximum value. (1)}
  Now look at this program.

  {\small
\begin{verbatim}
    if a > b:
        if a > c:
            return a
        else:
            return c
    else:
        if c > b:
            return c
        else:
            return b
\end{verbatim}
  }
  
  The author claims that this program takes three variables $a$, $b$
  and $c$ and returns the largest one.  \pause
  
  {\em Do you believe the author of the code? \pause Why?}
\end{frame}

\begin{frame}[fragile]\frametitle{Finding the maximum value. (2)}
  Finally, look at this program.

  {\small
\begin{verbatim}
  // Input: array A with n elements: A[0],...,A[n-1]
  m = 0
  for i = 0, 1, ..., n-1:
    if A[i] > m:
      m = A[i]
  return m  
\end{verbatim}
  }
  
  The author claims that this program takes an array $A$ with $n$
  elements and returns the maximum element.  \pause
  
  {\em Do you believe the author of the code? \pause Why?}
  \pause

  {\bf Can we try to test the code with all possible inputs?}
\end{frame}


\begin{frame}[fragile]\frametitle{Finding the maximum value. (3)}
  Let's try again.

  {\small
\begin{verbatim}
  // Input: array A with n elements: A[0],...,A[n-1]
  m = A[0]
  for i = 1, 2, ..., n-1:
    if A[i] > m:
      m = A[i]
  return m  
\end{verbatim}
  }
  
  {\em Do you believe the author of the code? \pause Why?}
  \pause

  {\bf Can we try to test the code with all possible inputs?}
\end{frame}

\begin{frame}[fragile]\frametitle{Another example: testing primes (1)}
  A {\em prime} is a natural number greater than 1 that has no
  positive divisors other 1 and itself.  E.g., 2,3,7,11 are primes.
  \pause
  
  {\small
\begin{verbatim}
  // Input: an integer n
  if n <= 1:
      return False
  i = 2
  while i <= n-1:
      if n is divisible by i:
          return False
      i = i + 1
  return True
\end{verbatim}
  }

  The code above checks if $n$ is a prime number.   How fast can it run?
  \pause

  Note that if $n$ is a prime number, the for-loop repeats for $n-2$
  times.  Thus, the running time is approximately proportional to
  $n$.
  \pause

  Can we do better?
\end{frame}

\begin{frame}[fragile]\frametitle{Another example: testing primes (2)}
  Consider the following code.
  
  {\small
\begin{verbatim}
  // Input: an integer n
  if n <= 1:
      return False
  let s = square root of n
  i = 2
  while i <= s:
      if n is divisible by i:
          return False
      i = i + 1
  return True
\end{verbatim}
  }

  How fast can it run? \pause Note that $s = \sqrt{n}$; therefore, it
  takes time approximately proportional to $\sqrt{n}$ to run.
  \pause

  Ok, it should be faster.  {\bf But is it correct?}

\end{frame}

\begin{frame}\frametitle{Informal arguments (1)}
  \begin{itemize}
  \item Let's try to argue that the faster algorithm works correctly.
    \pause

  \item
    Note that if $n$ is a prime number, the algorithm answers
    correctly. (Why?)  \pause

  \item Therefore, let's consider the case when $n$ is not prime
    (i.e., $n$ is a composite).  \pause

  \item If that's the case, $n$ has some positive divisor which is not
    $1$ or $n$.  Let's call this number $a$.  \pause

  \item Now, if $2\leq a\leq\sqrt{n}$, at some point during the
    execution of the algorithm, $i=a$ and $i$ should divides $n$; thus
    the algorithm correctly returns {\tt False}.  \pause

  \item
    Are we done?
  \end{itemize}
  
\end{frame}

\begin{frame}\frametitle{Informal arguments (2)}
  \begin{itemize}
  \item Recall that we are left with the case that (1) $n$ is not
    prime and (2) its positive divisor $a$ is larger than
    $\sqrt{n}$. \pause

  \item
    Let $b=n/a$.  Since $n$ and $a$ are positive integers and $a$
    divides $n$, $b$ is also a positive integer.
    \pause

  \item
    Note that if we can argue that $2\leq b\leq\sqrt{n}$, we are done.
    (why?)
    \pause

  \item
    How can we do that?
  \end{itemize}
  
\end{frame}

\begin{frame}\frametitle{Informal arguments (3): quick break}
  \begin{itemize}
  \item {\bf Original goal:} To show that the faster algorithm is
    correct.  \pause
    
  \item
    {\bf Current (sub) goal:} Consider a positive composite $n$ and its
    positive divisor $a$, where $a>\sqrt{n}$.  Let $b=n/a$.  We want
    to show that $2\leq b\leq\sqrt{n}$.
  \end{itemize}
  
\end{frame}

\end{document}
