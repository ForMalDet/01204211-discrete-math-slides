\documentclass{beamer}
\usefonttheme[onlymath]{serif}
\usepackage{fancybox}
\usepackage{tcolorbox}
\begin{document}

\newcommand{\lecturetitle}[1]{
  \title{01204211 Discrete Mathematics \\ #1}
  \author{Jittat Fakcharoenphol}
  \frame{\titlepage}
}

\lecturetitle{Lecture 1: Introduction}

\begin{frame}\frametitle{What is this course about?}
  This is a math class. \\
  \pause
  But, what is mathematics? \\
  \pause
  Ah... that's a philosopical question. \\
  \pause
  IMHO, mathematics is a mean to communicate {\em precise} ideas.
\end{frame}

\begin{frame}\frametitle{It's like learning a new language}
  \begin{itemize}
  \item Do you remember the time when you start learning English?
    \pause
  \item There are a few things you have to learn and get used to.
    \pause
  \item They might not make so much sense in the beginning, but over
    time, you will get comfortable with how the language is used.
    \pause
  \item As your knowledge of the language gets better, everything
    becomes more natural.  Learning a new language sometimes expands
    your view of the world.
    \pause
  \item I hope it is also true with this course.
  \end{itemize}
\end{frame}

\begin{frame}\frametitle{The goals of this course}
  There are two goals:
  \begin{itemize}
  \item To learn how to make mathematical arguments.
    \pause
  \item To learn various fundamental mathematical concepts that are
    very useful in computer science.
  \end{itemize}
\end{frame}

%%%%%%%%%%%% why

\begin{frame}[fragile]\frametitle{Why should we learn how to prove? (1)}
  \pause
  Look at this program.

  \begin{tcolorbox}
\begin{verbatim}
    if a > b:
        return a
    else:
        return b
\end{verbatim}
  \end{tcolorbox}
  
  The author claims that this program takes two variables $a$ and $b$
  and returns the larger one.
  \pause
  
  {\em Do you believe the author of the code? \pause Why?}
\end{frame}

\begin{frame}[fragile]\frametitle{Finding the maximum value. (1)}
  Now look at this program.

  \begin{tcolorbox}
  {\small
\begin{verbatim}
    if a > b:
        if a > c:
            return a
        else:
            return c
    else:
        if c > b:
            return c
        else:
            return b
\end{verbatim}
  }
  \end{tcolorbox}
  
  The author claims that this program takes three variables $a$, $b$
  and $c$ and returns the largest one.  \pause
  
  {\em Do you believe the author of the code? \pause Why?}
\end{frame}

\begin{frame}[fragile]\frametitle{Finding the maximum value. (2)}
  Finally, look at this program.

  \begin{tcolorbox}
  {\small
\begin{verbatim}
  // Input: array A with n elements: A[0],...,A[n-1]
  m = 0
  for i = 0, 1, ..., n-1:
    if A[i] > m:
      m = A[i]
  return m  
\end{verbatim}
  }
  \end{tcolorbox}
  
  The author claims that this program takes an array $A$ with $n$
  elements and returns the maximum element.  \pause
  
  {\em Do you believe the author of the code? \pause Why?}
  \pause

  {\bf Can we try to test the code with all possible inputs?}
\end{frame}


\begin{frame}[fragile]\frametitle{Finding the maximum value. (3)}
  Let's try again.

  \begin{tcolorbox}
  {\small
\begin{verbatim}
  // Input: array A with n elements: A[0],...,A[n-1]
  m = A[0]
  for i = 1, 2, ..., n-1:
    if A[i] > m:
      m = A[i]
  return m  
\end{verbatim}
  }
  \end{tcolorbox}
  
  {\em Do you believe the author of the code? \pause Why?}
  \pause

  {\bf Can we try to test the code with all possible inputs?}
\end{frame}

\begin{frame}[fragile]\frametitle{Another example: testing primes (1)}
  A {\em prime} is a natural number greater than 1 that has no
  positive divisors other 1 and itself.  E.g., 2,3,7,11 are primes.
  \pause

  \begin{tcolorbox}
  {\small
\begin{verbatim}
  Algorithm CheckPrime(n):     // Input: an integer n
      if n <= 1:
          return False
      i = 2
      while i <= n-1:
          if n is divisible by i:
              return False
          i = i + 1
      return True
\end{verbatim}
  }
  \end{tcolorbox}
  
  The code above checks if $n$ is a prime number.   How fast can it run?
  \pause

  Note that if $n$ is a prime number, the for-loop repeats for $n-2$
  times.  Thus, the running time is approximately proportional to
  $n$.
  \pause

  Can we do better?
\end{frame}

\begin{frame}[fragile]\frametitle{Another example: testing primes (2)}
  Consider the following code.
  
  \begin{tcolorbox}
  {\small
\begin{verbatim}
  Algorithm CheckPrime2(n):  // Input: an integer n
      if n <= 1:
          return False
      let s = square root of n
      i = 2
      while i <= s:
          if n is divisible by i:
              return False
          i = i + 1
      return True
\end{verbatim}
  }
  \end{tcolorbox}
  
  How fast can it run? \pause Note that $s = \sqrt{n}$; therefore, it
  takes time approximately proportional to $\sqrt{n}$ to run.
  \pause

  Ok, it should be faster.  {\bf But is it correct?}

\end{frame}

\begin{frame}\frametitle{Informal arguments (1)}
  \begin{itemize}
  \item Let's try to argue the Algorithm {\tt CheckPrime2} works
    correctly.  \pause

  \item
    Note that if $n$ is a prime number, the algorithm answers
    correctly. (Why?)  \pause

  \item Therefore, let's consider the case when $n$ is not prime
    (i.e., $n$ is a composite).  \pause

  \item If that's the case, $n$ has some positive divisor which is not
    $1$ or $n$.  Let's call this number $a$.  \pause

  \item Now, if $2\leq a\leq\sqrt{n}$, at some point during the
    execution of the algorithm, $i=a$ and $i$ should divides $n$; thus
    the algorithm correctly returns {\tt False}.  \pause

  \item
    Are we done?
  \end{itemize}
  
\end{frame}

\begin{frame}\frametitle{Informal arguments (2)}
  \begin{itemize}
  \item Recall that we are left with the case that (1) $n$ is not
    prime and (2) its positive divisor $a$ is larger than
    $\sqrt{n}$. \pause

  \item
    Let $b=n/a$.  Since $n$ and $a$ are positive integers and $a$
    divides $n$, $b$ is also a positive integer.
    \pause

  \item
    Note that if we can argue that $2\leq b\leq\sqrt{n}$, we are done.
    (why?)
    \pause

  \item
    How can we do that?
  \end{itemize}
  
\end{frame}

\begin{frame}\frametitle{The goals}
  \begin{itemize}
  \item Let's take a break and look back at what we are trying to do.
    \pause

    \begin{tcolorbox}
      {\bf Original goal:} To show that Algorithm {\tt CheckPrime2} is
      correct.
    \end{tcolorbox}
    \pause
    
    \begin{tcolorbox}
      {\bf Current (sub) goal:} Consider a positive composite $n$ and
      its positive divisor $a$, where $a>\sqrt{n}$.  Let $b=n/a$.  We
      want to show that $2\leq b\leq\sqrt{n}$.
    \end{tcolorbox}
    \pause

  \item Before we continue, I'd like to add a bit of formalism to our
    thinking process.
  \end{itemize}
\end{frame}

\begin{frame}\frametitle{The main goal}
  \begin{itemize}
  \item {\bf Original goal:} To show that Algorithm {\tt CheckPrime2}
    is correct.  \pause
  \item Let's focus on the statement we want to argue for:
    
    \begin{tcolorbox}
      {\bf ``Algorithm {\tt CheckPrime2} is correct.''}
    \end{tcolorbox}

    \pause
  \item
    Note that this statement can either be ``true'' or ``false.''  If
    we can demonstrate, using logical/mathematical arguments that this
    statement is true, we can say that we {\bf prove} the statement.
  \end{itemize}
\end{frame}

\begin{frame}\frametitle{The (sub) goal}
  \begin{itemize}
  \item
    {\bf Current (sub) goal:} Consider a positive composite $n$ and its
    positive divisor $a$, where $a>\sqrt{n}$.  Let $b=n/a$.  We want
    to show that $2\leq b\leq\sqrt{n}$.
    \pause
  \item Let's focus only on the statement we want to argue for:
    \pause

    \begin{tcolorbox}
      \begin{center}
        $2\leq b\leq\sqrt{n}$.
      \end{center}
    \end{tcolorbox}
    
    \pause
  \item If we only look at this statement, it is unclear if the
    statement is true or false because there are variables $b$ and $n$
    in the statement.  It can be true in some case and it can be false
    in some case depending on the values of $n$ and $b$.

    \pause

  \item Are we doom? \pause Not really.  The statement above is not
    precisely the statement we want to prove.
  \end{itemize}
\end{frame}

\begin{frame}\frametitle{The (sub) goal (second try)}
  \begin{itemize}
  \item
    {\small {\bf Current (sub) goal:} Consider a positive composite $n$ and its
    positive divisor $a$, where $a>\sqrt{n}$.  Let $b=n/a$.  We want
    to show that $2\leq b\leq\sqrt{n}$.}

  \item We can be more specific about what values of $n$ and $b$ that
    we want to consider. \pause

    \begin{tcolorbox}[title=Revised statement]
      For all positive composite integer $n$, and for some of its
      divisor $a$ such that $a > \sqrt{n}$,
      \[ 2\leq b\leq\sqrt{n},\]
      where $b=n/a$.
    \end{tcolorbox}

  \item Note that this revised statement is now ``quantified,'' that
    is, every variable in the statement has specific scope.  Now the
    statement is either true or false.
  \end{itemize}
\end{frame}

\begin{frame}\frametitle{Propositions}
  \begin{itemize}
  \item A {\em proposition} is a statement which is either {\bf true}
    or {\bf false}.
    \pause
  \item It is our basic unit of mathematical ``facts''.
    
  \item Examples:
    \begin{itemize}
    \item Algorithm {\tt CheckPrime2} is correct.
    \item $10^2 = 100$.
    \item $\sqrt{2}$ is irrational.
    \end{itemize}

    \pause
  \item Examples of statements which are not propositions (why?):
    \begin{itemize}
    \item $x > 10$.
    \item $1+2+\cdots+10$.
    \item This algorithm is fast.
    \item Run, run fast.
    \end{itemize}
  \end{itemize}
\end{frame}

\begin{frame}\frametitle{Combining propositions}
  \begin{itemize}
  \item We usually use a variable to refer to a proposition.  For
    example, we may use $P$ to stand for ``it rains'' or $Q$ to stand
    for ``the road is wet.''
    \pause
  \item The truth value of a variable is the truth value of the
    proposition it stands for.
    \pause
  \item Many propositions can be combined to get a complex statement
    using logical operators.  \pause
  \item For example, we can join $P$ and $Q$ with ``and'' (denoted by
    ``$\wedge$'') and get
    \[P\wedge Q,\]
    which stands for ``it rains and the road is wet''.
    \pause
    
  \item An expression $P\wedge Q$ is an example of {\em propositional
    forms}.  The logical value of a propositional form ``usually''
    depends on the truth value of its variables.
  \end{itemize}
\end{frame}

\begin{frame}\frametitle{Connectives: ``and'', ``or'', ``not''}
  Given propositions $P$ and $Q$, we can use connectives to form more
  complex propositions:
  \begin{itemize}
  \item {\bf Conjunction:} $P\wedge Q$ (``$P$ and $Q$''), \\
    (True when both $P$ and $Q$ are true)
  \item {\bf Disjunction:} $P\vee Q$ (``$P$ or $Q$''), \\
    (True when at least one of $P$ and $Q$ is true)
  \item {\bf Negation:} $\neg P$ (``not $P$'') \\
    (True only when $P$ is false)
  \end{itemize}
  \pause

  If $P$ stands for ``today is Tuesday'' and $Q$ stands for
  ``dogs are animals'', then
  \begin{itemize}
  \item $P\wedge Q$ stands for ``today is Tuesday and dogs are animals'',
  \item $P\vee Q$ stands for ``today is Tuesday or dogs are animals'', and
  \item $\neg P$ stands for ``today is not Tuesday''.
  \end{itemize}
\end{frame}

\begin{frame}\frametitle{Truth tables}
  To represents values of propositional forms, we usually use truth tables.
  \begin{tcolorbox}[title=And/Or/Not]
    \begin{tabular}{|c|c||c|c|c|}
      \hline
      $P$ & $Q$ & $P\wedge Q$ & $P\vee Q$ & $\neg P$ \\
      \hline
      $T$ & $T$ & $T$ & $T$ & $F$ \\
      $T$ & $F$ & $F$ & $T$ & \\
      $F$ & $T$ & $F$ & $T$ & $T$ \\
      $F$ & $F$ & $F$ & $F$ & \\
      \hline
    \end{tabular}
  \end{tcolorbox}
\end{frame}

\begin{frame}\frametitle{Quick check 1}
  For each of these statements, define propositional variables
  representing each proposition inside the statement and write the
  proposition form of the statement.
  \begin{itemize}
  \item All prime numbers are larger than 0 and all natural numbers is
    at least one.
  \item You are smart or you won't be taking this class.
  \end{itemize}
\end{frame}

\begin{frame}\frametitle{Quick check 2}
  Use a truth table to find the values of (1) $P\wedge \neg P$ and (2)
  $P\vee\neg P$.
  \pause

  \begin{tcolorbox}[title=And/Or/Not]
    \begin{tabular}{|c|c||c|c|}
      \hline
      $P$ & $\neg P$ & $P\wedge\neg P$ & $P\vee\neg P$\\
      \hline
      $T$ & $F$ & $F$ & $T$ \\
      $F$ & $T$ & $F$ & $T$ \\
      \hline
    \end{tabular}
  \end{tcolorbox}
  \pause

  Note that $P\wedge\neg P$ is always false and $P\vee\neg P$ is always true.

  A propositional form which is always true regardless of the truth
  values of its variables is called a {\em tautology.}  On the other
  hand, a propositional form which is always false regardless of the
  truth values of its variables is called a {\em contradiction.}
\end{frame}

\begin{frame}\frametitle{Implications}
  Given $P$ and $Q$, an implication
  \[P\Rightarrow Q\]
  stands for ``if $P$, then $Q$''.  This is a very important
  propositional form.

  It states that ``when $P$ is true, $Q$ must be true''.  Let's try to
  fill in its truth table:

  \begin{tcolorbox}[title=Implications]
    \begin{tabular}{|c|c||c|}
      \hline
      $P$ & $Q$ & $P\Rightarrow Q$ \\
      \hline
      $T$ & $T$ & \pause $T$ \\
      $T$ & $F$ & \pause $F$ \\
      $F$ & $T$ & \pause $T$ \\
      $F$ & $F$ & \pause $T$ \\
      \hline
    \end{tabular}
  \end{tcolorbox}
\end{frame}

\begin{frame}\frametitle{What?}
  \begin{itemize}
  \item
    Yes, when $P$ is false, $P\Rightarrow Q$ is {\bf always true} no
    matter what truth value of $Q$ is. 

  \item We say that in this case, the statement $P\Rightarrow Q$ is
    {\em vacuously true.}
    \pause

  \item
    You might feel a bit uncomfortable about this, because in most
    natural languages, when we say that if $P$, then $Q$ we sometimes
    mean something more than that in the logical expression
    ``$P\Rightarrow Q$.''
  \end{itemize}
\end{frame}

  
\begin{frame}\frametitle{One explanation}
  \begin{itemize}
  \item
    But let's look closely at what it means when we say that:

    \begin{tcolorbox}
      if $P$ is true, $Q$ must be true.
    \end{tcolorbox}

  \item
    Note that this statement does not say anything about the case when
    $P$ is false, i.e., it only considers the case when $P$ is true.
    \pause
    
  \item
    Therefore, having that $P\Rightarrow Q$ is true is OK with the
    case that (1) $Q$ is false when $P$ is false, and (2) $Q$ is true
    when $P$ is false.
    \pause
    
  \item
    This is an example when mathematical language is ``stricter'' than
    natural language.
  \end{itemize}
\end{frame}

\begin{frame}\frametitle{Noticing if-then}
  We can write ``if $P$, then $Q$'' for $P\Rightarrow Q$, but there
  are other ways to say this. E.g., we can write (1) $Q$ if $P$, (2) $P$
  only if $Q$, or (3) when $P$, then $Q$.

  \pause

  \begin{tcolorbox}[title=Quick check 3]
    For each of these statements, define
    propositional variables representing each proposition inside the
    statement and write the proposition form of the statement.
    \begin{itemize}
    \item If you do not have enough sleep, you will feel dizzy during class.
    \item If you eat a lot and you do not have enough exercise, you will
      get fat.
    \item You can get A from this course, only if you work fairly hard.
    \end{itemize}
  \end{tcolorbox}
  
\end{frame}

\begin{frame}\frametitle{Only-if}
  Let $P$ be ``you get A from this course.''

  Let $Q$ be ``you work fairly hard.''
  
  Let $R$ be ``You can get A from this course, only if you work fairly hard.''

  Let's think about the truth values of $R$.
  
  \begin{tcolorbox}[title=Only if you work fairly hard.]
    \begin{tabular}{|c|c||c|}
      \hline
      $P$ & $Q$ & $R$ \\
      \hline
      $T$ & $T$ & \\
      $T$ & $F$ & \\
      $F$ & $T$ & \\
      $F$ & $F$ & \\
      \hline
    \end{tabular}
  \end{tcolorbox}
  \pause

  Thus, $R$ should be logically equivalent to $P\Rightarrow Q$.  (We
  write $R\equiv P\Rightarrow Q$ in this case.)
\end{frame}

\begin{frame}\frametitle{If and only if: ($\Leftrightarrow$)}
  Given $P$ and $Q$, we denote by
  \[ P\Leftrightarrow Q \]
  the statement ``$P$ if and only if $Q$.''
  \pause
  It is logically equivalent to
  \[ (P\Leftarrow Q)\wedge (P\Rightarrow Q), \]
  i.e., $P\Leftrightarrow Q \equiv (P\Leftarrow Q)\wedge (P\Rightarrow Q)$.

  Let's fill in its truth table.
  \begin{tcolorbox}
    \begin{tabular}{|c|c||c|c|c|}
      \hline
      $P$ & $Q$ & $P\Rightarrow Q$ & $P\Leftarrow Q$ & $P\Leftrightarrow Q$ \\
      \hline
      $T$ & $T$ & & & \\
      $T$ & $F$ & & & \\
      $F$ & $T$ & & & \\
      $F$ & $F$ & & & \\
      \hline
    \end{tabular}
  \end{tcolorbox}
\end{frame}

\begin{frame}\frametitle{An implication and its friends}
  When you have two propositions
  \begin{itemize}
  \item $P$ = ``I own a cell phone'', and
  \item $Q$ = ``I bring a cell phone to class''.
  \end{itemize}
  We have
  \begin{itemize}
  \item an implication $P\Rightarrow Q$ $\equiv$ \\ ``If I own a cell phone,
    I'll bring it to class'',
  \item its {\bf converse} $Q\Rightarrow P$ $\equiv$ \\ ``If I bring a cell phone
    to class, I own it'', and
  \item its {\bf contrapositive} $\neg Q\Rightarrow\neg P$ $\equiv$ \\ ``If I do not
    bring a cell phone to class, I do not own one''.
  \end{itemize}
\end{frame}

\begin{frame}\frametitle{Quick check 4}
  Let's consider the following truth table:
  \begin{tcolorbox}
    \begin{tabular}{|c|c||c|c|c|}
      \hline
      $P$ & $Q$ & $P\Rightarrow Q$ & $Q\Rightarrow P$ & $\neg Q \Rightarrow \neg P$ \\
      \hline
      $T$ & $T$ & & & \\
      $T$ & $F$ & & & \\
      $F$ & $T$ & & & \\
      $F$ & $F$ & & & \\
      \hline
    \end{tabular}
  \end{tcolorbox}
  \pause
  Do you notice any equivalence?
\end{frame}

\end{document}
